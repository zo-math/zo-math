\documentclass[12pt]{article} % Định dạng tài liệu, kích thước chữ 12pt

\usepackage{polyglossia} % Quản lí ngôn ngữ
\setdefaultlanguage{vietnamese}
\setotherlanguages{english}
\usepackage{fontspec} % Cung cấp khả năng sử dụng phông chữ OpenType và TrueType
\usepackage{
    amsmath, % Các lệnh toán học
    amsfonts, % Các kí hiệu toán học
    amssymb % Các kí hiệu toán học
}
\usepackage{unicode-math} % Cung cấp hỗ trợ cho các phông chữ toán học Unicode
\setmainfont{STIX Two Text} % Thiết lập phông chữ chính là STIX Two Text
\setmathfont{STIX Two Math} % Thiết lập phông chữ toán học là STIX Two Math

\usepackage[a4paper, left=30mm, right=20mm, top=25mm, bottom=25mm]{geometry} % Định dạng kích thước và lề trang

\usepackage{graphicx} % Hỗ trợ chèn hình ảnh vào tài liệu

\usepackage{xcolor} % Gói màu sắc để tùy chỉnh màu sắc

\usepackage[vietnamese]{hyperref} % Tạo liên kết và tham chiếu trong tài liệu
\hypersetup{
    colorlinks=true, % Kích hoạt màu sắc cho các liên kết
    linkcolor=darkgray, % Màu của liên kết nội bộ
    citecolor=blue, % Màu của liên kết tham chiếu
    filecolor=blue, % Màu của liên kết tập tin
    urlcolor=blue % Màu của liên kết URL
}
\renewcommand{\sectionautorefname}{Mục} % Đổi tên tự động của các liên kết phần mục từ "Section" thành "Mục"

\usepackage{bookmark} % Tạo mục lục nhanh và chính xác hơn

\usepackage[backend=biber,style=authoryear,sorting=none]{biblatex} % Quản lí tài liệu tham khảo với biblatex và biber
\addbibresource{references.bib} % Thêm tài liệu tham khảo từ tệp references.bib
\DeclareLanguageMapping{vietnamese}{vietnamese-english} % Định nghĩa ánh xạ ngôn ngữ cho tiếng Việt
\DeclareLanguageMapping{english}{english} % Định nghĩa ánh xạ ngôn ngữ cho tiếng Anh

\newcounter{myproblem} % Tạo một bộ đếm mới cho môi trường myproblem
\newenvironment{myproblem}[1][]{%
    \vspace{10pt} % Khảng cách từ trên xuống
    \noindent\textsc{Bài tập #1} % Định dạng tiêu đề môi trường myproblem
    \noindent
}{%
    \par
    \vspace{10pt} % Khoảng cách từ dưới lên
}
\newenvironment{mydotproblem}[1][]{%
    \vspace{10pt}
    \noindent\textsc{Bài tập. #1}
    \noindent
}{%
    \par
    \vspace{10pt} 
}
\newenvironment{mynumproblem}[1][]{%
    \refstepcounter{myproblem}
    \vspace{10pt}
    \noindent\textsc{Bài tập \theproblem. #1}
    \noindent
}{%
    \par
    \vspace{10pt} 
}

\title{Tiêu đề}
\author{Nguyễn Tấn Nhựt}
\date{\today}

\allowdisplaybreaks % Cho phép ngắt dòng trong các công thức toán học dài

\begin{document}

\maketitle
% \tableofcontents

\section{Sự kiện ngẫu nhiên}

% Bạn muốn dự đoán kết quả chạm đất khi tung một đồng xu. Trước tiên, bạn cần lí tưởng hóa thí nghiệm này bằng cách loại bỏ các kết quả không đáng quan tâm như đôi khi đồng xu có thể lăn đi hoặc đứng trên cạnh của nó, và những yếu tố khó kiểm soát như gió, lực tung, hoặc đồng xu không hoàn toàn đối xứng cũng có thể ảnh hưởng đến kết quả. Chúng ta cùng đồng ý chỉ xét đến hai kết quả sấp và ngửa là khả thi của thí nghiệm này. 

% Sự lý tưởng hóa này là một quy ước thường thấy trong khoa học, giúp làm cho lý thuyết dễ tiếp cận hơn mà vẫn giữ được tính ứng dụng thực tế. Khi không có yếu tố ưu tiên cho một trong hai mặt, chúng ta có thể gán xác suất bằng nhau cho mỗi kết quả, và xác suất của mỗi mặt là 1/2.

Trong thế giới hiện thực, khi tung một đồng xu, lúc chạm đất, kết quả có thể là mặt sấp hoặc mặt ngửa, đôi khi đồng xu có thể lăn đi hoặc đứng trên cạnh của nó. Cùng với những yếu tố khó kiểm soát như gió, lực tung, hoặc sự không đối xứng của đồng xu, có thể ảnh hưởng đến kết quả.

Trong thế giới lí tưởng, khi tung một đồng xu, lúc chạm đất, kết quả chỉ có thể là mặt sấp hoặc mặt ngửa, 

Để dự đoán kết quả khi tung một đồng xu, trước tiên, chúng ta cần lý tưởng hóa thí nghiệm này. Điều này bao gồm việc loại bỏ các kết quả không đáng quan tâm, chẳng hạn như trường hợp đồng xu có thể lăn đi hoặc đứng trên cạnh, cùng với những yếu tố khó kiểm soát như gió, lực tung, hoặc sự không đối xứng của đồng xu, có thể ảnh hưởng đến kết quả. Do đó, chúng ta sẽ chỉ xem xét hai kết quả khả thi là mặt sấp và mặt ngửa.

Sự lý tưởng hóa này là một quy ước phổ biến trong khoa học, giúp đơn giản hóa lý thuyết mà vẫn đảm bảo tính ứng dụng thực tế.  

Khi không có yếu tố nào ưu tiên cho một trong hai mặt, chúng ta có thể gán xác suất bằng nhau cho mỗi kết quả, với xác suất của mỗi mặt là 1/2.

Bất kỳ lý thuyết nào cũng đòi hỏi phải có sự lý tưởng hóa, và lý tưởng hóa đầu tiên của chúng ta liên quan đến các kết quả có thể của một thí nghiệm hoặc quan sát. Nếu chúng ta muốn xây dựng một mô hình trừu tượng, trước hết chúng ta phải đưa ra quyết định về những gì tạo thành một kết quả khả thi của thí nghiệm (đã được lý tưởng hóa).

Để sử dụng thuật ngữ đồng nhất, các kết quả của thí nghiệm hoặc quan sát sẽ được gọi là sự kiện. Do đó, chúng ta sẽ nói về sự kiện khi tung năm đồng xu có hơn ba đồng rơi vào mặt ngửa. Tương tự, thí nghiệm "phân chia các lá bài trong trò chơi bridge" có thể dẫn đến sự kiện "người chơi phía Bắc có hai con át chủ." Thành phần của một mẫu ("hai người thuận tay trái trong mẫu 85 người") và kết quả của một phép đo ("nhiệt độ 120 độ," "bảy đường dây điện thoại bận") đều sẽ được gọi là sự kiện.

Chúng ta sẽ phân biệt giữa các sự kiện hợp (hoặc có thể phân tích) và sự kiện đơn (hoặc không thể phân tích). Ví dụ, nói rằng kết quả của việc tung hai con xúc xắc là "tổng bằng sáu" tương đương với việc nói rằng kết quả là "(1, 5) hoặc (2, 4) hoặc (3, 3) hoặc (4, 2) hoặc (5, 1)", và danh sách này phân tích sự kiện "tổng bằng sáu" thành năm sự kiện đơn. Tương tự, sự kiện "hai mặt lẻ" có thể được phân tích thành "(1, 1) hoặc (1, 3) hoặc ... hoặc (5, 5)" thành chín sự kiện đơn. Lưu ý rằng nếu kết quả là (3, 3), thì lần tung xúc xắc đó cũng đồng thời dẫn đến sự kiện "tổng bằng sáu" và "hai mặt lẻ"; các sự kiện này không loại trừ lẫn nhau và do đó có thể xảy ra đồng thời.

\noindent\rule{\textwidth}{.4pt}

Chúng ta sẽ bắt đầu từ những trải nghiệm đơn giản nhất, chẳng hạn như tung đồng xu hoặc gieo xúc xắc, nơi mà tất cả các tuyên bố đều có ý nghĩa trực quan rõ ràng. Sự trực quan này sẽ được chuyển thành một mô hình trừu tượng để dần dần tổng quát hóa. Các ví dụ minh họa sẽ được cung cấp để giải thích nền tảng thực nghiệm của nhiều mô hình khác nhau và phát triển trực giác của người đọc, nhưng bản thân lý thuyết sẽ mang tính chất toán học. Chúng ta sẽ không cố gắng giải thích "ý nghĩa thực sự" của xác suất, cũng như nhà vật lý hiện đại không phân tích kỹ "ý nghĩa thực sự" của khối lượng và năng lượng, hoặc nhà hình học không bàn về bản chất của một điểm. Thay vào đó, chúng ta sẽ chứng minh các định lý và chỉ ra cách chúng được áp dụng.

\noindent\rule{\textwidth}{.4pt}

Ngay lúc đó và tại đó, bạn ghi nhận được một kết quả đã xảy ra. Nhưng cũng ngay lúc đó và tại đó, nhiều kết quả khác cũng có thể đã xảy ra thay cho kết quả đó, và trước khi bạn biết kết quả nào sẽ xảy ra, chúng được gọi là các \emph{sự kiện ngẫu nhiên}.

Ra mặt sấp khi bạn tung một đồng xu là một sự kiện ngẫu nhiên. Số bé trai được sinh ra trong một năm là một sự kiện ngẫu nhiên. 

Tung đồng xu là một \emph{thí nghiệm}. Thí nghiệm là một quá trình được thiết kế, trong đó người thực hiện có thể kiểm soát và điều chỉnh các điều kiện để kiểm tra một giả thuyết cụ thể. Mỗi lần tung đồng xu là một \emph{phép thử} của thí nghiệm. Tương tự, gieo xúc xắc cũng là một thí nghiệm, với mỗi lần gieo là một phép thử. kết quả của mỗi phép thử có thể là một trong sáu sự kiện ngẫu nhiên: mặt một chấm, hai chấm, ba chấm, bốn chấm, năm chấm, hoặc sáu chấm.

Số bé trai được sinh ra trong một năm lại là một \emph{quan sát}. Quan sát là quá trình thu thập dữ liệu từ các hiện tượng mà không can thiệp hay kiểm soát các yếu tố đầu vào. Thay vì tạo ra các điều kiện, người quan sát chỉ ghi nhận các kết quả xảy ra. Tương tự, số lượng bệnh nhân hôm nay đến khám ở một bệnh viện là một quan sát. 

Với các hiện tượng như số bé trai được sinh ra, mỗi lần quan sát thường không diễn ra trong cùng điều kiện như trước. Ví dụ, dân số mỗi năm đều thay đổi, và do đó điều kiện quan sát cũng không đồng nhất như trong các thí nghiệm có phép thử lặp lại.

Khi gieo xúc xắc, sự kiện ra mặt một chấm là một \emph{sự kiện đơn}, trong khi sự kiện ra mặt chẵn là một \emph{sự kiện hợp}, vì nó bao gồm ba sự kiện đơn: mặt hai chấm, mặt bốn chấm và mặt sáu chấm. Tương tự, trong quan sát về số bé trai sinh ra, sự kiện có nhiều hơn 50 bé trai sinh ra trong một năm cũng là một sự kiện hợp, vì nó bao gồm nhiều khả năng khác nhau về số lượng bé trai.

Một sự kiện đơn tương ứng với một kết quả cụ thể. Một sự kiện hợp là tập hợp của nhiều sự kiện đơn. Sự kiện đơn thì không thể phân tích được nữa, trong khi sự kiện hợp có thể phân tích thành các sự kiện đơn.

\noindent\rule{\textwidth}{.4pt}

Lý thuyết toán học về xác suất - có giá trị thực tiễn và ý nghĩa trực quan trong mối liên hệ với các thí nghiệm thực tế hoặc khái niệm như tung một đồng xu một lần, tung một đồng xu 100 lần, ném ba viên xúc xắc, sắp xếp một bộ bài, ghép hai bộ bài lại với nhau, chơi roulette, quan sát tuổi thọ của một nguyên tử phóng xạ hoặc một người, chọn một mẫu ngẫu nhiên người và quan sát số lượng người thuận tay trái trong đó, lai ghép hai loài thực vật và quan sát kiểu hình của thế hệ con; hoặc với các hiện tượng như giới tính của một em bé mới sinh, số lượng đường dây bận trong một tổng đài điện thoại, số cuộc gọi trên một điện thoại, tiếng ồn ngẫu nhiên trong một hệ thống thông tin điện tử, kiểm soát chất lượng định kỳ của một quy trình sản xuất, tần suất tai nạn, số lượng sao đôi trong một khu vực trên bầu trời, vị trí của một hạt dưới sự khuếch tán. Tất cả các mô tả này đều khá mơ hồ, và để làm cho lý thuyết trở nên có ý nghĩa, chúng ta phải thống nhất về những gì chúng ta muốn nói đến khi nói về các kết quả khả thi của thí nghiệm hoặc quan sát đang được đề cập.

Khi một đồng xu được tung, nó không nhất thiết phải rơi xuống mặt "đầu" hoặc "đuôi"; nó có thể lăn đi hoặc đứng trên cạnh. Tuy nhiên, chúng ta sẽ đồng ý coi "đầu" và "đuôi" là hai kết quả khả thi duy nhất của thí nghiệm. Quy ước này làm đơn giản hóa lý thuyết mà không ảnh hưởng đến khả năng ứng dụng của nó. Những sự lý tưởng hóa như vậy là thực hành tiêu chuẩn. Không thể đo lường tuổi thọ của một nguyên tử hay một con người mà không có sai số, nhưng vì mục đích lý thuyết, việc hình dung rằng những số lượng này là những con số chính xác là hợp lý. Câu hỏi đặt ra là số nào thực sự có thể đại diện cho tuổi thọ của một con người. Có một độ tuổi tối đa nào mà sau đó cuộc sống là không thể, hay bất kỳ độ tuổi nào cũng có thể xảy ra? Chúng ta ngần ngại thừa nhận rằng con người có thể sống 1000 năm, nhưng thực tiễn hiện tại không đặt ra giới hạn nào cho khả năng sống. Theo các công thức mà các bảng tử vong hiện đại dựa vào, tỷ lệ người sống sót sau 1000 năm là khoảng một trong $10^{10^{36}}$ - một con số có $10^{27}$ tỷ số không. Phát biểu này không có ý nghĩa từ góc độ sinh học hay xã hội học, nhưng nếu chỉ xem xét từ góc độ thống kê, nó không mâu thuẫn với bất kỳ kinh nghiệm nào. Trong một thế kỷ, có ít hơn $10^{10}$ người được sinh ra. Để kiểm tra lập luận này một cách thống kê, cần hơn $10^{10^{35}}$ thế kỷ, con số này lớn hơn nhiều so với $10^{10^{34}}$ tuổi thọ của trái đất. Rõ ràng, những xác suất cực kỳ nhỏ như vậy tương thích với khái niệm về sự không thể. Việc sử dụng chúng có thể có vẻ hoàn toàn vô lý, nhưng nó không gây hại và rất tiện lợi trong việc đơn giản hóa nhiều công thức. Hơn nữa, nếu chúng ta thực sự loại bỏ khả năng sống 1000 năm, chúng ta sẽ phải chấp nhận sự tồn tại của tuổi tối đa, và giả định rằng có thể sống $x$ năm và không thể sống $x$ năm và hai giây là một ý tưởng không hấp dẫn như khái niệm về cuộc sống không giới hạn.

Bất kỳ lý thuyết nào cũng đòi hỏi phải có sự lý tưởng hóa, và lý tưởng hóa đầu tiên của chúng ta liên quan đến các kết quả có thể của một thí nghiệm hoặc quan sát. Nếu chúng ta muốn xây dựng một mô hình trừu tượng, trước hết chúng ta phải đưa ra quyết định về những gì tạo thành một kết quả khả thi của thí nghiệm (đã được lý tưởng hóa).

Để sử dụng thuật ngữ đồng nhất, các kết quả của thí nghiệm hoặc quan sát sẽ được gọi là sự kiện. Do đó, chúng ta sẽ nói về sự kiện khi tung năm đồng xu có hơn ba đồng rơi vào mặt ngửa. Tương tự, thí nghiệm "phân chia các lá bài trong trò chơi bridge" có thể dẫn đến sự kiện "người chơi phía Bắc có hai con át chủ." Thành phần của một mẫu ("hai người thuận tay trái trong mẫu 85 người") và kết quả của một phép đo ("nhiệt độ 120 độ," "bảy đường dây điện thoại bận") đều sẽ được gọi là sự kiện.

Chúng ta sẽ phân biệt giữa các sự kiện hợp (hoặc có thể phân tích) và sự kiện đơn (hoặc không thể phân tích). Ví dụ, nói rằng kết quả của việc tung hai con xúc xắc là "tổng bằng sáu" tương đương với việc nói rằng kết quả là "(1, 5) hoặc (2, 4) hoặc (3, 3) hoặc (4, 2) hoặc (5, 1)", và danh sách này phân tích sự kiện "tổng bằng sáu" thành năm sự kiện đơn. Tương tự, sự kiện "hai mặt lẻ" có thể được phân tích thành "(1, 1) hoặc (1, 3) hoặc ... hoặc (5, 5)" thành chín sự kiện đơn. Lưu ý rằng nếu kết quả là (3, 3), thì lần tung xúc xắc đó cũng đồng thời dẫn đến sự kiện "tổng bằng sáu" và "hai mặt lẻ"; các sự kiện này không loại trừ lẫn nhau và do đó có thể xảy ra đồng thời.

\section{Không gian mẫu}

Không gian mẫu là tập hợp tất cả các kết cục có thể xảy ra của một thí nghiệm hay một quan sát. Khi tung một đồng xu, không gian mẫu gồm hai kết cục: mặt sấp và mặt ngửa. Khi gieo một xúc xắc, không gian mẫu bao gồm sáu kết cục, tương ứng với các mặt của xúc xắc.

Không gian mẫu đóng vai trò quan trọng vì nó giúp chúng ta hiểu rõ và xác định tất cả những gì có thể xảy ra trong một tình huống ngẫu nhiên. Khi có một không gian mẫu, các sự kiện ngẫu nhiên có thể được định nghĩa dưới dạng tập hợp con của không gian này. Ví dụ, khi gieo xúc xắc, sự kiện ra mặt chẵn có thể được mô tả là tập hợp ${2, 4, 6}$.

Một điểm cần lưu ý là không gian mẫu luôn bao gồm toàn bộ các khả năng, do đó, không gian mẫu chính là \emph{sự kiện chắc chắn}. Điều này có nghĩa là khi ta thực hiện thí nghiệm hoặc quan sát, chắc chắn một trong những kết cục trong không gian mẫu sẽ xảy ra, dù ta không biết trước đó sẽ là kết cục nào.

Sự xuất hiện của không gian mẫu giúp lý thuyết xác suất trở nên mạch lạc và nhất quán hơn. Khi đã biết các sự kiện là tập hợp con của không gian mẫu, ta có thể tiến đến việc gán xác suất cho chúng. Và vì không gian mẫu biểu thị toàn bộ khả năng, việc gán xác suất từ 0 đến 1 cho các sự kiện trở nên tự nhiên: xác suất 1 ứng với sự kiện chắc chắn (toàn bộ không gian mẫu), và xác suất 0 ứng với sự kiện không thể xảy ra (tập hợp rỗng).

\section{Xác suất}
Đối tượng nghiên cứu của lí thuyết xác suất là các hình mẫu ngẫu nhiên có hai tính chất:
\begin{description}
    \item [Không chắc chắn về kết quả:] Chúng ta không biết chính xác kết quả trước khi sự kiện xảy ra tạo thành kết cục nhưng có thể mô tả bằng các khả năng xảy ra khác nhau.
    \item [Tính qui luật trong dài hạn:] Mặc dù kết quả của một lần xảy ra có vẻ ngẫu nhiên, nhưng khi sự kiện được lặp đi lặp lại nhiều lần, sẽ có một mô hình hay xu hướng có thể quan sát được. Xác suất giúp mô tả xu hướng đó. 
\end{description}

\end{document}
