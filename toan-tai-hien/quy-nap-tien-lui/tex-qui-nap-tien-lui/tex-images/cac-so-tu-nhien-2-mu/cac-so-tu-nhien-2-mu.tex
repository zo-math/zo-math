\documentclass[12pt]{standalone}

\usepackage{polyglossia} % Quản lí ngôn ngữ
\setdefaultlanguage{vietnamese}
\setotherlanguages{english}
\usepackage{fontspec} % Cung cấp khả năng sử dụng phông chữ OpenType và TrueType
\usepackage{
    amsmath, % Các lệnh toán học
    amsfonts, % Các kí hiệu toán học
    amssymb, % Các kí hiệu toán học
    amsthm % Môi trường định lí
}
\usepackage{unicode-math} % Cung cấp hỗ trợ cho các phông chữ toán học Unicode
\setmainfont{STIX Two Text} % Thiết lập phông chữ chính là STIX Two Text
\setmathfont{STIX Two Math} % Thiết lập phông chữ toán học là STIX Two Math
\usepackage{tikz}

\begin{document}

\begin{tikzpicture}
    \foreach \i in {1,...,20} {
        % Vị trí x của ô vuông, kích thước mỗi ô là 0.5 đơn vị
        \pgfmathsetmacro\x{(\i - 1) * 0.75}
        % Vẽ các ô vuông
        \draw (\x, 0) rectangle (\x+0.5, 0.5);
        % Tô màu đen cho các ô đặc biệt
        \ifnum\i=1 \fill[black] (\x, 0) rectangle (\x+0.5, 0.5); \fi
        \ifnum\i=2 \fill[black] (\x, 0) rectangle (\x+0.5, 0.5); \fi
        \ifnum\i=4 \fill[black] (\x, 0) rectangle (\x+0.5, 0.5); \fi
        \ifnum\i=8 \fill[black] (\x, 0) rectangle (\x+0.5, 0.5); \fi
        \ifnum\i=16 \fill[black] (\x, 0) rectangle (\x+0.5, 0.5); \fi
        % Thêm số ở phía trên các ô vuông
        \node at (\x+0.25, 0.75) {\i};
        \node at (20*.75+0.25, 0.75) {\dots};
        \node at (20*.75+0.25, 0.25) {\dots};
      }
\end{tikzpicture}

\end{document}
