\documentclass[12pt]{article} % Định dạng tài liệu, kích thước chữ 12pt

\usepackage{polyglossia} % Quản lí ngôn ngữ
\setdefaultlanguage{vietnamese}
\setotherlanguages{english}
\usepackage{fontspec} % Cung cấp khả năng sử dụng phông chữ OpenType và TrueType
\usepackage{
    amsmath, % Các lệnh toán học
    amsfonts, % Các kí hiệu toán học
    amssymb % Các kí hiệu toán học
}
\usepackage{unicode-math} % Cung cấp hỗ trợ cho các phông chữ toán học Unicode
\setmainfont{STIX Two Text} % Thiết lập phông chữ chính là STIX Two Text
\setmathfont{STIX Two Math} % Thiết lập phông chữ toán học là STIX Two Math

\usepackage[a4paper, left=2cm, right=2cm, top=2cm, bottom=2cm]{geometry} % Định dạng kích thước và lề trang

\usepackage{graphicx} % Hỗ trợ chèn hình ảnh vào tài liệu

\usepackage{xcolor} % Gói màu sắc để tùy chỉnh màu sắc

\usepackage[vietnamese]{hyperref} % Tạo liên kết và tham chiếu trong tài liệu
\hypersetup{
    colorlinks=true, % Kích hoạt màu sắc cho các liên kết
    linkcolor=darkgray, % Màu của liên kết nội bộ
    citecolor=blue, % Màu của liên kết tham chiếu
    filecolor=blue, % Màu của liên kết tập tin
    urlcolor=blue % Màu của liên kết URL
}
\renewcommand{\sectionautorefname}{Mục} % Đổi tên tự động của các liên kết phần mục từ "Section" thành "Mục"

\usepackage{bookmark} % Tạo mục lục nhanh và chính xác hơn

\usepackage[backend=biber,style=authoryear,sorting=none]{biblatex} % Quản lí tài liệu tham khảo với biblatex và biber
\addbibresource{references.bib} % Thêm tài liệu tham khảo từ tệp references.bib
\DeclareLanguageMapping{vietnamese}{vietnamese-english} % Định nghĩa ánh xạ ngôn ngữ cho tiếng Việt
\DeclareLanguageMapping{english}{english} % Định nghĩa ánh xạ ngôn ngữ cho tiếng Anh

\newcounter{myproblem} % Tạo một bộ đếm mới cho môi trường myproblem
\newenvironment{myproblem}[1][]{%
    \vspace{10pt} % Khảng cách từ trên xuống
    \noindent\textsc{Bài tập #1} % Định dạng tiêu đề môi trường myproblem
    \noindent
}{%
    \par
    \vspace{10pt} % Khoảng cách từ dưới lên
}
\newenvironment{mydotproblem}[1][]{%
    \vspace{10pt}
    \noindent\textsc{Bài tập. #1}
    \noindent
}{%
    \par
    \vspace{10pt} 
}
\newenvironment{mynumproblem}[1][]{%
    \refstepcounter{myproblem}
    \vspace{10pt}
    \noindent\textsc{Bài tập \theproblem. #1}
    \noindent
}{%
    \par
    \vspace{10pt} 
}

\title{Khám phá không gian đa chiều: Xu hướng $90^\circ$}
\author{Nguyễn Tấn Nhựt}
\date{}

\allowdisplaybreaks % Cho phép ngắt dòng trong các công thức toán học dài

\begin{document}

\maketitle
% \tableofcontents
Cho $u$ và $u_j$ là các véc-tơ đơn vị, hỏi kích thước trung bình của các tích vô hướng $\lvert u\cdot u_j\rvert$ là gì. Theo giải tích, trung bình là $\frac{1}{\pi}\int_0^\pi\lvert\cos(\theta)\rvert d\theta=\frac{2}{\pi}$.

Bên trên là bài toán 28 ở trang 17 trong sách Introduction to Linear Algebra (ấn bản thứ 6) của Gilbert Strang xuất bản năm 2023. Nguyên văn: Using $v=\mathrm{randn(3,1)}$ in MATLAB, create a random unit vector $u=v/\lVert v\rVert$. Using $V=\mathrm{randn(3,30)}$ create 30 more random unit vectors $U_j$. What is the average size of the dot products $\lvert u\cdot U_j\rvert$? In calculus, the average is $\int_0^\pi\lvert\cos\theta\rvert d\theta/\pi=2/\pi$.

Hiểu bài toán trên:
\begin{itemize}
    \item Hàm randn(3,1) sinh ra một véc-tơ 3 thành phần theo phân phối chuẩn.
    \item Véc-tơ $u=v/\lVert v\rVert$ là véc-tơ đơn vị, nó được xem là có phân phối đều trên hình cầu đơn vị.
    \item Các véc-tơ đơn vị được hiểu là lấy ngẫu nhiên trên hình cầu đơn vị.
    \item Trung bình của $\lvert u\cdot u_j\rvert$ hội tụ về $\frac{1}{\pi}\int_0^\pi\lvert\cos(\theta)\rvert d\theta=\frac{2}{\pi}$.
    \item Trung bình lí thuyết của $\lvert\cos(\theta)\rvert$ khi góc $\theta$ phân bố đều từ $0$ đến $\pi$ là $\frac{2}{\pi}$. 
\end{itemize}
Khi giải bài toán này tôi buộc phải nghĩ về:
\begin{itemize}
    \item Thay vì các véc-tơ 3D, có thể thử với 2D, 4D, 5D, và vân vân.
    \item Số chiều cao thì tốt hay thấp thì tốt?
    \item Trong không gian nhiều chiều các véc-tơ ngẫu nhiên có xu hướng gần như vuông góc với nhau do độ phân tán tăng lên. Điều này xuất phát từ sự tăng trưởng nhanh chóng của không gian các véc-tơ có thể chiếm. Điều này làm giảm giá trị trung bình của $\lvert u\cdot U_j\rvert$ khi số chiều tăng, khác với kết quả lí thuyết $\frac{2}{\pi}$.
    \item Góc $\theta$ trong trường hợp 2D hay 3D là quen thuộc với chúng ta. Với số chiều cao hơn, tuy chúng ta không thể hình dung ra được, nhưng qua bài toán này chúng ta có thể hiểu thêm ý nghĩa của nó trong không gian đa chiều nói chung.
\end{itemize}

\noindent\rule{\linewidth}{.4pt} \\

Hai véc-tơ ngẫu nhiên được lấy ra từ không gian có xu hướng vuông góc nhau khi số chiều tăng lên. Đây là một hiện tượng thú vị có liên quan đến lí thuyết xác suất và hình học trong không gian đa chiều.

Khi làm việc trong các không gian có số chiều lớn, một hiện tượng đáng chú ý xuất hiện: hai véc-tơ được chọn ngẫu nhiên có xu hướng vuông góc nhau khi số chiều tăng.

Đây không chỉ là hệ quả của cách chúng ta định nghĩa vuông góc trong không gian nhiều chiều, mà còn phản ánh bản chất hình học của không gian khi số chiều tăng lên. 

Trong không gian 2 hay 3 chiều chúng ta dễ dàng hình dung hai véc-tơ có thể tạo ra một góc bất kì. Tuy nhiên khi số chiều của không gian tăng lên, góc của 2 véc-tơ ngẫu nhiên có xu hướng tiến gần đến $90^\circ$. Điều này có nghĩa là 2 véc-tơ gần như vuông góc nhau khi số chiều rất lớn.

Khi số chiều tăng lên, xác suất tích vô hướng của 2 véc-tơ ngẫu nhiên tiến dần về 0, dần đến $\cos(\theta)$ gần bằng 0, tức là góc $\theta$ gần bằng $90^\circ$. Hiện tượng này xảy ra do sự phân bố của các véc-tơ trong không gian cao chiều, nơi mà phần lớn không gian nằm gàn các mặt phẳng vuông góc.

"Hiệu ứng tập chung khối lượng" trong không gian $n$ chiều.

$n=4$

[0.12753114 0.80740502 0.52795282 0.27456068 0.54539886 0.88011085 0.45093717 0.47791611 0.35881778 0.2611181] \\

$n=5$

[0.10332762 0.28315885 0.52423106 0.77848733 0.29068516 0.07151115 0.46362643 0.20022192 0.17443497 0.34728237] \\

\end{document}
