% Options for packages loaded elsewhere
\PassOptionsToPackage{unicode}{hyperref}
\PassOptionsToPackage{hyphens}{url}
\PassOptionsToPackage{dvipsnames,svgnames,x11names}{xcolor}
%
\documentclass[
  letterpaper,
  DIV=11,
  numbers=noendperiod]{scrartcl}

\usepackage{amsmath,amssymb}
\usepackage{iftex}
\ifPDFTeX
  \usepackage[T1]{fontenc}
  \usepackage[utf8]{inputenc}
  \usepackage{textcomp} % provide euro and other symbols
\else % if luatex or xetex
  \usepackage{unicode-math}
  \defaultfontfeatures{Scale=MatchLowercase}
  \defaultfontfeatures[\rmfamily]{Ligatures=TeX,Scale=1}
\fi
\usepackage{lmodern}
\ifPDFTeX\else  
    % xetex/luatex font selection
\fi
% Use upquote if available, for straight quotes in verbatim environments
\IfFileExists{upquote.sty}{\usepackage{upquote}}{}
\IfFileExists{microtype.sty}{% use microtype if available
  \usepackage[]{microtype}
  \UseMicrotypeSet[protrusion]{basicmath} % disable protrusion for tt fonts
}{}
\makeatletter
\@ifundefined{KOMAClassName}{% if non-KOMA class
  \IfFileExists{parskip.sty}{%
    \usepackage{parskip}
  }{% else
    \setlength{\parindent}{0pt}
    \setlength{\parskip}{6pt plus 2pt minus 1pt}}
}{% if KOMA class
  \KOMAoptions{parskip=half}}
\makeatother
\usepackage{xcolor}
\setlength{\emergencystretch}{3em} % prevent overfull lines
\setcounter{secnumdepth}{-\maxdimen} % remove section numbering
% Make \paragraph and \subparagraph free-standing
\makeatletter
\ifx\paragraph\undefined\else
  \let\oldparagraph\paragraph
  \renewcommand{\paragraph}{
    \@ifstar
      \xxxParagraphStar
      \xxxParagraphNoStar
  }
  \newcommand{\xxxParagraphStar}[1]{\oldparagraph*{#1}\mbox{}}
  \newcommand{\xxxParagraphNoStar}[1]{\oldparagraph{#1}\mbox{}}
\fi
\ifx\subparagraph\undefined\else
  \let\oldsubparagraph\subparagraph
  \renewcommand{\subparagraph}{
    \@ifstar
      \xxxSubParagraphStar
      \xxxSubParagraphNoStar
  }
  \newcommand{\xxxSubParagraphStar}[1]{\oldsubparagraph*{#1}\mbox{}}
  \newcommand{\xxxSubParagraphNoStar}[1]{\oldsubparagraph{#1}\mbox{}}
\fi
\makeatother


\providecommand{\tightlist}{%
  \setlength{\itemsep}{0pt}\setlength{\parskip}{0pt}}\usepackage{longtable,booktabs,array}
\usepackage{calc} % for calculating minipage widths
% Correct order of tables after \paragraph or \subparagraph
\usepackage{etoolbox}
\makeatletter
\patchcmd\longtable{\par}{\if@noskipsec\mbox{}\fi\par}{}{}
\makeatother
% Allow footnotes in longtable head/foot
\IfFileExists{footnotehyper.sty}{\usepackage{footnotehyper}}{\usepackage{footnote}}
\makesavenoteenv{longtable}
\usepackage{graphicx}
\makeatletter
\newsavebox\pandoc@box
\newcommand*\pandocbounded[1]{% scales image to fit in text height/width
  \sbox\pandoc@box{#1}%
  \Gscale@div\@tempa{\textheight}{\dimexpr\ht\pandoc@box+\dp\pandoc@box\relax}%
  \Gscale@div\@tempb{\linewidth}{\wd\pandoc@box}%
  \ifdim\@tempb\p@<\@tempa\p@\let\@tempa\@tempb\fi% select the smaller of both
  \ifdim\@tempa\p@<\p@\scalebox{\@tempa}{\usebox\pandoc@box}%
  \else\usebox{\pandoc@box}%
  \fi%
}
% Set default figure placement to htbp
\def\fps@figure{htbp}
\makeatother

% Cấu hình font chữ chính và toán học
\usepackage{fontspec}
\usepackage{unicode-math}
\setmainfont{STIX Two Text}
\setsansfont{STIX Two Text}
\setmonofont{STIX Two Text}
\setmathfont{STIX Two Math}

% Cấu hình kích thước chữ
% \fontsize{12pt}{14pt}\selectfont

% Đánh số section
\setcounter{secnumdepth}{3}

% Các gói hỗ trợ bổ sung
\usepackage{hyperref}  % Hỗ trợ liên kết
\usepackage{graphicx}  % Hỗ trợ hình ảnh
\usepackage{amsmath, amssymb}  % Hỗ trợ toán học nâng cao
\usepackage{svg}  % Hỗ trợ hình ảnh SVG
\usepackage{xcolor}
\usepackage{tikz}
  \usetikzlibrary{arrows, positioning}

\usepackage{booktabs}
\usepackage{longtable}
\usepackage{array}
\usepackage{multirow}
\usepackage{wrapfig}
\usepackage{float}
\usepackage{colortbl}
\usepackage{pdflscape}
\usepackage{tabu}
\usepackage{threeparttable}
\usepackage{threeparttablex}
\usepackage[normalem]{ulem}
\usepackage{makecell}
\usepackage{xcolor}
\KOMAoption{captions}{tableheading}
\makeatletter
\@ifpackageloaded{tcolorbox}{}{\usepackage[skins,breakable]{tcolorbox}}
\@ifpackageloaded{fontawesome5}{}{\usepackage{fontawesome5}}
\definecolor{quarto-callout-color}{HTML}{909090}
\definecolor{quarto-callout-note-color}{HTML}{0758E5}
\definecolor{quarto-callout-important-color}{HTML}{CC1914}
\definecolor{quarto-callout-warning-color}{HTML}{EB9113}
\definecolor{quarto-callout-tip-color}{HTML}{00A047}
\definecolor{quarto-callout-caution-color}{HTML}{FC5300}
\definecolor{quarto-callout-color-frame}{HTML}{acacac}
\definecolor{quarto-callout-note-color-frame}{HTML}{4582ec}
\definecolor{quarto-callout-important-color-frame}{HTML}{d9534f}
\definecolor{quarto-callout-warning-color-frame}{HTML}{f0ad4e}
\definecolor{quarto-callout-tip-color-frame}{HTML}{02b875}
\definecolor{quarto-callout-caution-color-frame}{HTML}{fd7e14}
\makeatother
\makeatletter
\@ifpackageloaded{caption}{}{\usepackage{caption}}
\AtBeginDocument{%
\ifdefined\contentsname
  \renewcommand*\contentsname{Table of contents}
\else
  \newcommand\contentsname{Table of contents}
\fi
\ifdefined\listfigurename
  \renewcommand*\listfigurename{List of Figures}
\else
  \newcommand\listfigurename{List of Figures}
\fi
\ifdefined\listtablename
  \renewcommand*\listtablename{List of Tables}
\else
  \newcommand\listtablename{List of Tables}
\fi
\ifdefined\figurename
  \renewcommand*\figurename{Figure}
\else
  \newcommand\figurename{Figure}
\fi
\ifdefined\tablename
  \renewcommand*\tablename{Table}
\else
  \newcommand\tablename{Table}
\fi
}
\@ifpackageloaded{float}{}{\usepackage{float}}
\floatstyle{ruled}
\@ifundefined{c@chapter}{\newfloat{codelisting}{h}{lop}}{\newfloat{codelisting}{h}{lop}[chapter]}
\floatname{codelisting}{Listing}
\newcommand*\listoflistings{\listof{codelisting}{List of Listings}}
\makeatother
\makeatletter
\makeatother
\makeatletter
\@ifpackageloaded{caption}{}{\usepackage{caption}}
\@ifpackageloaded{subcaption}{}{\usepackage{subcaption}}
\makeatother

\ifLuaTeX
\usepackage[bidi=basic]{babel}
\else
\usepackage[bidi=default]{babel}
\fi
\babelprovide[main,import]{vietnamese}
% get rid of language-specific shorthands (see #6817):
\let\LanguageShortHands\languageshorthands
\def\languageshorthands#1{}
\usepackage{bookmark}

\IfFileExists{xurl.sty}{\usepackage{xurl}}{} % add URL line breaks if available
\urlstyle{same} % disable monospaced font for URLs
\hypersetup{
  pdftitle={Bài 1. Các số đặc trưng đo xu thế trung tâm cho mẫu số liệu ghép nhóm},
  pdfauthor={ZO \textbar{} 2025-03-26},
  pdflang={vi},
  colorlinks=true,
  linkcolor={blue},
  filecolor={Maroon},
  citecolor={Blue},
  urlcolor={Blue},
  pdfcreator={LaTeX via pandoc}}


\title{Bài 1. Các số đặc trưng đo xu thế trung tâm cho mẫu số liệu ghép
nhóm}
\usepackage{etoolbox}
\makeatletter
\providecommand{\subtitle}[1]{% add subtitle to \maketitle
  \apptocmd{\@title}{\par {\large #1 \par}}{}{}
}
\makeatother
\subtitle{Chương 5 \textbar{} Cánh Diều \textbar{} Toán 11}
\author{ZO \textbar{} 2025-03-26}
\date{}

\begin{document}
\maketitle


Nếu không có yêu cầu làm tròn đặc biệt, các con số sẽ được làm tròn đến
4 chữ số thập phân để tiện đối chiếu khi bạn tự tìm đáp án.

\section{Bài học}\label{buxe0i-hux1ecdc}

\section*{Hoạt động 1}

Trong \emph{Bảng 1}, ta thấy:

\begin{itemize}
\tightlist
\item
  Có 13 ô-tô có độ tuổi dưới 4;
\item
  Có 29 ô-tô có độ tuổi từ 4 đến dưới 8.
\end{itemize}

Hãy xác định số ô-tô có độ tuổi:

\begin{itemize}
\tightlist
\item
  Từ 8 đến dưới 12;
\item
  Từ 12 đến dưới 16;
\item
  Từ 16 đến dưới 20.
\end{itemize}

\begin{longtable*}{cc}
\toprule
Nhóm & Tần số\\
\midrule
\endfirsthead
\multicolumn{2}{@{}l}{\textit{(continued)}}\\
\toprule
Nhóm & Tần số\\
\midrule
\endhead

\endfoot
\bottomrule
\endlastfoot
\([0;4)\) & 13\\
\([4;8)\) & 29\\
\([8;12)\) & 48\\
\([12;16)\) & 22\\
\([16;20)\) & 8\\
\addlinespace
 & \(n=120\)\\*
\end{longtable*}

\begin{center}
\textbf{Lời giải}
\end{center}

\begin{tcolorbox}[enhanced jigsaw, opacityback=0, opacitybacktitle=0.6, colback=white, left=2mm, arc=.35mm, toprule=.15mm, title={Mẹo}, bottomtitle=1mm, coltitle=black, breakable, bottomrule=.15mm, rightrule=.15mm, colbacktitle=quarto-callout-note-color!10!white, leftrule=.75mm, colframe=quarto-callout-note-color-frame, toptitle=1mm, titlerule=0mm]

Bạn có biết? Khi chưa quen với cách lập luận, bắt chước sách giáo khoa
là một cách tốt để làm quen với cách trình bày. Khi đã hiểu rõ, bạn có
thể tự phát triển cách lập luận của riêng mình.

\end{tcolorbox}

Từ \emph{Bảng 1}, quan sát theo hàng để đếm số ô-tô trong từng nhóm độ
tuổi.

\begin{itemize}
\tightlist
\item
  Có 48 ô-tô có độ tuổi từ 8 đến dưới 12.
\item
  Có 22 ô-tô có độ tuổi từ 12 đến dưới 16.
\item
  Có 8 ô-tô có độ tuổi từ 16 đến dưới 20.
\end{itemize}

\begin{tcolorbox}[enhanced jigsaw, opacityback=0, opacitybacktitle=0.6, colback=white, left=2mm, arc=.35mm, toprule=.15mm, title={Nhóm \& tần số}, bottomtitle=1mm, coltitle=black, breakable, bottomrule=.15mm, rightrule=.15mm, colbacktitle=quarto-callout-note-color!10!white, leftrule=.75mm, colframe=quarto-callout-note-color-frame, toptitle=1mm, titlerule=0mm]

Việc xác định có 13 ô-tô có độ tuổi dưới 4 cho mình biết rằng nhóm
\([0;4)\) có tần số 13.

\begin{itemize}
\tightlist
\item
  \textbf{Nhóm}: \([a; b)\), chứa một số giá trị của mẫu số liệu, \(a\)
  là \emph{đầu mút trái}, \(b\) là \emph{đầu mút phải}. \emph{Độ dài}
  nhóm là \(b-a\).
\item
  \textbf{Tần số}: số giá trị của mẫu số liệu có trong nhóm \([a; b)\).
\end{itemize}

Biết có 13 ô-tô có độ tuổi dưới 4, nhưng không biết tuổi cụ thể của từng
ô-tô.

\end{tcolorbox}

\begin{tcolorbox}[enhanced jigsaw, opacityback=0, opacitybacktitle=0.6, colback=white, left=2mm, arc=.35mm, toprule=.15mm, title={Khám phá}, bottomtitle=1mm, coltitle=black, breakable, bottomrule=.15mm, rightrule=.15mm, colbacktitle=quarto-callout-note-color!10!white, leftrule=.75mm, colframe=quarto-callout-note-color-frame, toptitle=1mm, titlerule=0mm]

Nhìn vào bảng số liệu, bạn có nhận xét gì về số lượng ô-tô theo từng
nhóm độ tuổi?

\begin{itemize}
\tightlist
\item
  Số lượng ô-tô cũ (trên 12 năm) có ít hơn xe mới không?
\item
  Điều này có ý nghĩa gì?
\end{itemize}

\end{tcolorbox}

\section*{Luyện tập 1}

Mẫu số liệu ghép nhóm ở \emph{Bảng 1} có bao nhiêu số liệu? Bao nhiêu
nhóm? Tìm tần số của mỗi nhóm.

\begin{longtable*}{cc}
\toprule
Nhóm & Tần số\\
\midrule
\endfirsthead
\multicolumn{2}{@{}l}{\textit{(continued)}}\\
\toprule
Nhóm & Tần số\\
\midrule
\endhead

\endfoot
\bottomrule
\endlastfoot
\([0;4)\) & 13\\
\([4;8)\) & 29\\
\([8;12)\) & 48\\
\([12;16)\) & 22\\
\([16;20)\) & 8\\
\addlinespace
 & \(n=120\)\\*
\end{longtable*}

\begin{center}
\textbf{Lời giải}
\end{center}

\begin{itemize}
\tightlist
\item
  Mẫu số liệu ghép nhóm ở \emph{Bảng 1} có 120 số liệu, chính là tổng
  của cột Tần số.
\item
  Mẫu số liệu này có 5 nhóm, bao gồm \([0;4)\), \([4;8)\), \([8;12)\),
  \([12;16)\) và \([16;20)\). Để ngắn gọn, gọi các nhóm này lần lượt là
  1, 2, 3, 4 và 5.
\item
  Tần số của các nhóm 1, 2, 3, 4, 5 lần lượt là 13, 29, 48, 22, 8.
\end{itemize}

\section*{Hoạt động 2}

Một trường trung học phổ thông chọn 36 học sinh nam của khối lớp 11, đo
chiều cao của các bạn học sinh đó và thu được mẫu số liệu sau (đơn vị:
cm):

\begin{table}[!h]
\centering
\begin{tabular}{cccccccccccc}
\toprule
160 & 161 & 161 & 162 & 162 & 162 & 163 & 163 & 163 & 164 & 164 & 164\\
164 & 165 & 165 & 165 & 165 & 165 & 166 & 166 & 166 & 166 & 167 & 167\\
168 & 168 & 168 & 168 & 169 & 169 & 170 & 171 & 171 & 172 & 172 & 174\\
\bottomrule
\end{tabular}
\end{table}

Từ mẫu số liệu không ghép nhóm trên, hãy ghép các số liệu thành năm nhóm
theo các nửa khoảng có độ dài bằng nhau.

\begin{center}
Lời giải
\end{center}

\begin{tcolorbox}[enhanced jigsaw, opacityback=0, breakable, colback=white, left=2mm, rightrule=.15mm, leftrule=.75mm, arc=.35mm, colframe=quarto-callout-note-color-frame, bottomrule=.15mm, toprule=.15mm]

\vspace{-3mm}\textbf{Nhận thức quan trọng}\vspace{3mm}

Ở Hoạt động 1, chúng mình làm việc với một bảng tần số ghép nhóm đã có
sẵn, nhưng không biết chính xác các số liệu gốc.

Sang Hoạt động 2, tình huống đảo ngược: số liệu cụ thể được cho, nhưng
bảng tần số thì chưa có. Nhiệm vụ của chúng mình là ghép các số liệu
thành năm nhóm có độ dài bằng nhau.

Điều thú vị là, với hai tiêu chí ``số lượng nhóm'' và ``độ dài nhóm'',
mình chưa thể xác lập một cách ghép nhóm duy nhất. Vẫn còn nhiều cách
khác nhau để chia nhóm, và đây chính là cơ hội để khám phá sâu hơn về về
cách tổ chức và sắp xếp số liệu.

\begin{quote}
\begin{itemize}
\tightlist
\item
  Hoạt động 1: Có bảng tần số nhưng thiếu số liệu.
\item
  Hoạt động 2: Có số liệu nhưng cần tự ghép nhóm, và việc ghép nhóm
  không có duy nhất một cách làm.
\end{itemize}
\end{quote}

\end{tcolorbox}

Số liệu nhỏ nhất là 160, số liệu lớn nhất là 174, nên miền chứa toàn bộ
số liệu là \([160; 174]\). Cần chia miền này thành 5 nhóm có độ dài bằng
nhau, nên độ dài mỗi nhóm là \[
    \frac{174-160}{5}=2,8.
\]

Nhưng nếu chia nhóm theo độ dài 2,8 cm, sẽ gặp các ranh giới thập phân
không đẹp:

\([160;162,8)\), \([162,8;165,6)\), \([165,6;168,4)\),
\([168,4;171,2)\), \([171,2;174]\).

Rõ ràng, các số thập phân như 162,8 rất khó đọc, khó vẽ biểu đồ và không
phù hợp với cách làm thống kê trong thực tế.

Số 3 là số nguyên nhỏ nhất gần với 2,8. Nếu chọn độ dài nhóm là 3, miền
dữ liệu cần mở rộng ra một chút để bao phủ toàn bộ dữ liệu cũ. Lúc này,
nếu vẫn giữ đầu dưới là 160, thì đầu trên là \[
    160+5\cdot 3 = 175.
\]

Khi đó, các nhóm mới là

\([160;163)\), \([163;166)\), \([166;169)\), \([169;172)\),
\([172;175)\).

Nhóm cuối, \([172; 175)\), ghi theo dạng nửa khoảng là hợp lý, vì 175
không có trong số liệu gốc.

\begin{tcolorbox}[enhanced jigsaw, opacityback=0, opacitybacktitle=0.6, colback=white, left=2mm, arc=.35mm, toprule=.15mm, title={Khám phá}, bottomtitle=1mm, coltitle=black, breakable, bottomrule=.15mm, rightrule=.15mm, colbacktitle=quarto-callout-note-color!10!white, leftrule=.75mm, colframe=quarto-callout-note-color-frame, toptitle=1mm, titlerule=0mm]

Có điều gì ngăn cản mình chọn ghép các số liệu thành năm nhóm sau đây
không:

\((159;162]\), \((162;165]\), \((165;168]\), \((168;171]\),
\((171;174]\);

hoặc

\([159;162)\), \([162;165)\), \([165;168)\), \([168;171)\),
\([171;174]\)?

Nếu không, câu hỏi kế tiếp là có bao nhiêu cách ghép nhóm? Cách nào là
tốt nhất và dựa trên tiêu chí nào để khẳng định?

\end{tcolorbox}

\section*{Luyện tập 2}

Một thư viện thống kê số người đến đọc sách vào buổi tối trong 30 ngày
của tháng vừa qua như sau:

\begin{table}[!h]
\centering
\begin{tabular}{cccccccccc}
\toprule
85 & 81 & 65 & 58 & 47 & 30 & 51 & 92 & 85 & 42\\
55 & 37 & 31 & 82 & 63 & 33 & 44 & 93 & 77 & 57\\
44 & 74 & 63 & 67 & 46 & 73 & 52 & 53 & 47 & 35\\
\bottomrule
\end{tabular}
\end{table}

Lập bảng tần số ghép nhóm có tám nhóm ứng với tám nửa khoảng sau:
\([25; 34)\), \([34; 43)\), \([43; 52)\), \([52; 61)\), \([61; 70)\),
\([70; 79)\), \([79 ; 88)\), \([88; 97)\).

\begin{center}
\textbf{Lời giải}
\end{center}

\begin{tcolorbox}[enhanced jigsaw, opacityback=0, breakable, colback=white, left=2mm, rightrule=.15mm, leftrule=.75mm, arc=.35mm, colframe=quarto-callout-note-color-frame, bottomrule=.15mm, toprule=.15mm]

\vspace{-3mm}\textbf{Mẹo}\vspace{3mm}

Sắp xếp số liệu theo thứ tự \emph{tăng dần} giúp mình dễ phân nhóm hơn.

\end{tcolorbox}

Để đảm bảo phân nhóm chính xác, số liệu nên được sắp xếp theo thứ tự
tăng dần.

\begin{longtable*}{cccccccccc}
\toprule
\endfirsthead
\multicolumn{10}{@{}l}{\textit{(continued)}}\\
\toprule
\endhead

\endfoot
\bottomrule
\endlastfoot
30 & 31 & 33 & 35 & 37 & 42 & 44 & 44 & 46 & 47\\
47 & 51 & 52 & 53 & 55 & 57 & 58 & 63 & 63 & 65\\
67 & 73 & 74 & 77 & 81 & 82 & 85 & 85 & 92 & 93\\*
\end{longtable*}

Nhìn vào dãy có thứ tự tăng dần, dễ dàng thấy rằng thuộc vào nhóm
\([25;34)\) có ba số liệu bao gồm 30; 31; 33. Bảng tần số ghép nhóm được
thành lập dựa trên các quan sát như thế cho đến khi duyệt hết số liệu.

\begin{longtable*}{cc}
\toprule
Nhóm & Tần số\\
\midrule
\endfirsthead
\multicolumn{2}{@{}l}{\textit{(continued)}}\\
\toprule
Nhóm & Tần số\\
\midrule
\endhead

\endfoot
\bottomrule
\endlastfoot
\([25;34)\) & 3\\
\([34;43)\) & 3\\
\([43;52)\) & 6\\
\([52;61)\) & 5\\
\([61;70)\) & 4\\
\addlinespace
\([70;79)\) & 3\\
\([79;88)\) & 4\\
\([88;97)\) & 2\\
 & \(n=30\)\\*
\end{longtable*}

\section*{Hoạt động 3}

Trong \emph{Bảng 4}, có bao nhiêu số liệu với giá trị không vượt quá giá
trị đầu mút phải:

\begin{enumerate}
\def\labelenumi{\alph{enumi}.}
\tightlist
\item
  163 của nhóm 1?
\item
  166 của nhóm 2?
\end{enumerate}

\begin{enumerate}
\def\labelenumi{\alph{enumi}.}
\setcounter{enumi}{2}
\tightlist
\item
  169 của nhóm 3?
\item
  172 của nhóm 4?
\end{enumerate}

\begin{enumerate}
\def\labelenumi{\alph{enumi}.}
\setcounter{enumi}{4}
\tightlist
\item
  175 của nhóm 5?
\end{enumerate}

\begin{longtable*}{cc}
\toprule
Nhóm & Tần số\\
\midrule
\endfirsthead
\multicolumn{2}{@{}l}{\textit{(continued)}}\\
\toprule
Nhóm & Tần số\\
\midrule
\endhead

\endfoot
\bottomrule
\endlastfoot
\([160;163)\) & 6\\
\([163;166)\) & 12\\
\([166;169)\) & 10\\
\([169;172)\) & 5\\
\([172;175)\) & 3\\
\addlinespace
 & \(n=36\)\\*
\end{longtable*}

\begin{center}
\textbf{Lời giải}
\end{center}

\begin{enumerate}
\def\labelenumi{\alph{enumi}.}
\tightlist
\item
  Nhóm 1, \([160; 163)\), có tần số là 6. Vậy có 6 số liệu không vượt
  quá 163.
\item
  Nhóm 2, \([163; 166)\), có tần số là 12. Các số liệu trong nhóm 1 cũng
  không vượt quá 166. Vậy tổng số là \[
  6+12=18.
  \]
\item
  Tương tự, số số liệu không vượt quá đầu mút phải 169 của nhóm 3 là \[
  6+12+10=28.
  \]
\item
  Số số liệu không vượt quá đầu mút phải 172 của nhóm 4 là \[
  6+12+10+5=33.
  \]
\item
  Số số liệu không vượt quá đầu mút phải 175 của nhóm 5 là \[
  6+12+10+5+3=36.
  \]
\end{enumerate}

\begin{tcolorbox}[enhanced jigsaw, opacityback=0, breakable, colback=white, left=2mm, rightrule=.15mm, leftrule=.75mm, arc=.35mm, colframe=quarto-callout-note-color-frame, bottomrule=.15mm, toprule=.15mm]

\vspace{-3mm}\textbf{Tần số tích lũy}\vspace{3mm}

Đáp số câu a cho biết tần số tích lũy của nhóm 1 là \(cf_1=6\). Đáp số
câu b cho biết tần số tích lũy của nhóm 2 là \(cf_2=18\). Đáp số câu e
cho biết tần số tích lũy của nhóm 5 là \(cf_5=36\).

\emph{Tần số tích lũy} của một nhóm là số số liệu có giá trị nhỏ hơn đầu
mút phải của nhóm đó.

\end{tcolorbox}

\section*{Luyện tập 3}

Trong bài toán ở Luyện tập 2, lập bảng tần số ghép nhóm bao gồm cả tần
số tích lũy có tám nhóm ứng với tám nửa khoång: \([25; 34)\),
\([34; 43)\), \([43; 52)\), \([52; 61)\), \([61; 70)\), \([70; 79)\),
\([79 ; 88)\), \([88; 97)\).

\begin{center}
\textbf{Lời giải}
\end{center}

\begin{tcolorbox}[enhanced jigsaw, opacityback=0, breakable, colback=white, left=2mm, rightrule=.15mm, leftrule=.75mm, arc=.35mm, colframe=quarto-callout-note-color-frame, bottomrule=.15mm, toprule=.15mm]

\vspace{-3mm}\textbf{Mẹo}\vspace{3mm}

Có hai cách để tính tần số tích lũy \(cf_3\) của nhóm 3.

\begin{itemize}
\tightlist
\item
  Cách 1: \(cf_3=n_1+n_2+n_3\), tính tần số tích lũy của nhóm 3 bằng
  cách cộng tần số của nhóm 1, nhóm 2 và nhóm 3.
\item
  Cách 2: \(cf_3=cf_2+n_3\), tính tần số tích lũy của nhóm 3 bằng cách
  cộng tần số tích lũy của nhóm 2 và tần số của nhóm 3.
\end{itemize}

Theo bạn, khi nào cách 2 nhanh hơn?

\end{tcolorbox}

\begin{longtable*}{ccc}
\toprule
Nhóm & Tần số & Tần số tích lũy\\
\midrule
\endfirsthead
\multicolumn{3}{@{}l}{\textit{(continued)}}\\
\toprule
Nhóm & Tần số & Tần số tích lũy\\
\midrule
\endhead

\endfoot
\bottomrule
\endlastfoot
\([25;34)\) & 3 & 3\\
\([34;43)\) & 3 & 6\\
\([43;52)\) & 6 & 12\\
\([52;61)\) & 5 & 17\\
\([61;70)\) & 4 & 21\\
\addlinespace
\([70;79)\) & 3 & 24\\
\([79;88)\) & 4 & 28\\
\([88;97)\) & 2 & 30\\
 & \(n=30\) & \\*
\end{longtable*}

\section*{Hoạt động 4}

Xét mẫu số liệu được cho trong \emph{Bảng 4}.

\begin{longtable*}{cc}
\toprule
Nhóm & Tần số\\
\midrule
\endfirsthead
\multicolumn{2}{@{}l}{\textit{(continued)}}\\
\toprule
Nhóm & Tần số\\
\midrule
\endhead

\endfoot
\bottomrule
\endlastfoot
\([160;163)\) & 6\\
\([163;166)\) & 12\\
\([166;169)\) & 10\\
\([169;172)\) & 5\\
\([172;175)\) & 3\\
\addlinespace
 & \(n=36\)\\*
\end{longtable*}

\begin{enumerate}
\def\labelenumi{\alph{enumi}.}
\tightlist
\item
  Tìm trung điểm \(x_1\) của nửa khoảng (tính bằng trung bình cộng của
  hai đầu mút) ứng với nhóm 1. Ta gọi trung điểm \(x_1\) là \emph{giá
  trị đại diện} của nhóm 1.
\item
  Bằng cách tương tự, hãy tìm giá trị đại diện của bốn nhóm còn lại. Từ
  đó, hãy hoàn thiện các số liệu trong \emph{Bảng 7}.
\item
  Tính giá trị \(\overline{x}\) cho bởi công thức sau: \[
  \overline{x} = \frac{n_1x_1+n_2x_2+\cdots+n_5x_5}{n}.
  \] Giá trị \(\overline{x}\) được gọi là \emph{số trung bình cộng} của
  muẫ số liệu đã cho.
\end{enumerate}

\begin{longtable*}{ccc}
\toprule
Nhóm & Giá trị đại diện & Tần số\\
\midrule
\endfirsthead
\multicolumn{3}{@{}l}{\textit{(continued)}}\\
\toprule
Nhóm & Giá trị đại diện & Tần số\\
\midrule
\endhead

\endfoot
\bottomrule
\endlastfoot
\([160;163)\) & \(x_1=?\) & \(n_1=?\)\\
\([163;166)\) & \(x_2=?\) & \(n_2=?\)\\
\([166;169)\) & \(x_3=?\) & \(n_3=?\)\\
\([169;172)\) & \(x_4=?\) & \(n_4=?\)\\
\([172;175)\) & \(x_5=?\) & \(n_5=?\)\\
\addlinespace
 &  & \(n=36\)\\*
\end{longtable*}

\begin{center}
\textbf{Lời giải}
\end{center}

\begin{enumerate}
\def\labelenumi{\alph{enumi}.}
\item
  Giá trị đại diện của nhóm 1, \([160; 163)\), là \[
       x_1=\frac{160+163}{2}=161,5 \text{ (cm)}.
   \]
\item
  Tương tự, giá trị đại diện cho các nhóm còn lại được tính như dưới
  đây.

  \begin{itemize}
  \tightlist
  \item
    Nhóm 2, \([163; 166)\): \(x_2=\frac{163+166}{2}=164,5\) (cm).
  \item
    Nhóm 3, \([166; 169)\): \(x_3=\frac{166+169}{2}=167,5\) (cm).
  \item
    Nhóm 4, \([169; 172)\): \(x_4=\frac{169+172}{2}=170,5\) (cm).
  \item
    Nhóm 5, \([172; 175)\): \(x_5=\frac{172+175}{2}=173,5\) (cm).
  \end{itemize}
\item
  Từ \emph{Bảng 4} và các giá trị đại diện vừa tính, \emph{Bảng 7} được
  hoàn thiện như dưới đây.
\end{enumerate}

\begin{longtable*}{ccc}
\toprule
Nhóm & Giá trị đại diện & Tần số\\
\midrule
\endfirsthead
\multicolumn{3}{@{}l}{\textit{(continued)}}\\
\toprule
Nhóm & Giá trị đại diện & Tần số\\
\midrule
\endhead

\endfoot
\bottomrule
\endlastfoot
\([160;163)\) & \(x_1=161,5\) & \(n_1=6\)\\
\([163;166)\) & \(x_2=164,5\) & \(n_2=12\)\\
\([166;169)\) & \(x_3=167,5\) & \(n_3=10\)\\
\([169;172)\) & \(x_4=170,5\) & \(n_4=5\)\\
\([172;175)\) & \(x_5=173,5\) & \(n_5=3\)\\
\addlinespace
 &  & \(n=36\)\\*
\end{longtable*}

Số trung bình cộng của mẫu số liệu là \begin{align*}
        \overline{x} 
            & = \frac{1}{36}(6\cdot 161,5+12\cdot 164,5+10\cdot 167,5+5\cdot170,5+3\cdot 173,5) \\
            & = \frac{1997}{12} \\
            & \approx 166,4167 \text{ (cm)}.
    \end{align*}

\section*{Luyện tập 4}

Xác định số trung bình cộng của mẫu số liệu ghép nhóm trong bài toán ở
Luyện tập 2.

\begin{center}
\textbf{Lời giải}
\end{center}

Bảng tần số ghép nhóm và giá trị đại diện của mẫu số liệu trong bài toán
ở Luyện tập 2 được cho trong bảng dưới đây.

\begin{longtable*}{ccc}
\toprule
Nhóm & Giá trị đại diện & Tần số\\
\midrule
\endfirsthead
\multicolumn{3}{@{}l}{\textit{(continued)}}\\
\toprule
Nhóm & Giá trị đại diện & Tần số\\
\midrule
\endhead

\endfoot
\bottomrule
\endlastfoot
\([25;34)\) & 29,5 & 3\\
\([34;43)\) & 38,5 & 3\\
\([43;52)\) & 47,5 & 6\\
\([52;61)\) & 56,5 & 5\\
\([61;70)\) & 65,5 & 4\\
\addlinespace
\([70;79)\) & 74,5 & 3\\
\([79;88)\) & 83,5 & 4\\
\([88;97)\) & 92,5 & 2\\
 &  & \(n=30\)\\*
\end{longtable*}

Trung bình cộng là \begin{align*}
    \overline{x}
        & = \frac{1}{30}(3\cdot 29,5 + 3\cdot 38,5 + 6\cdot 47,5 + 5\cdot 56,5 + 4\cdot 65,5 + 3\cdot 74,5 + 4\cdot 83,5 + 2\cdot 92,5) \\
        & = \frac{296}{5} \\
        & = 59,2 \text{ (người)}.
\end{align*}

\section*{Hoạt động 5}

Trong phòng thí nghiệm, người ta chia 99 mẫu vật thành năm nhóm căn cứ
trên khối lượng của chúng (đơn vị: g) và lập bảng tần số ghép nhóm bao
gồm cả tần số tích luỹ như \emph{Bảng 10}.

\begin{enumerate}
\def\labelenumi{\alph{enumi}.}
\tightlist
\item
  Nhóm 3 là nhóm đầu tiên có tần số tích luỹ lớn hơn hoặc bằng \[
   \frac{n}{2}=\frac{99}{2}=49,5 \text{ có đúng không?}
  \]
\item
  Tìm đầu mút trái \(r\), độ dài \(d\), tần số \(n_3\) của nhóm 3 và tần
  số tích lũy \(cf_2\) của nhóm 2.
\item
  Tính giá trị \(M_e\) theo công thức sau: \[
   M_e = r + \left (\frac{49,5-cf_2}{n_3}\right)\cdot d.
  \] Giá trị \(M_e\) được gọi là \emph{trung vị} của mẫu số liệu ghép
  nhóm đã cho.
\end{enumerate}

\begin{longtable*}{ccc}
\toprule
Nhóm & Tần số & Tần số tích lũy\\
\midrule
\endfirsthead
\multicolumn{3}{@{}l}{\textit{(continued)}}\\
\toprule
Nhóm & Tần số & Tần số tích lũy\\
\midrule
\endhead

\endfoot
\bottomrule
\endlastfoot
\([27,5;32,5)\) & 16 & 16\\
\([32,5;37,5)\) & 24 & 40\\
\([37,5;42,5)\) & 20 & 60\\
\([42,5;47,5)\) & 30 & 90\\
\([47,5;52,5)\) & 9 & 99\\
\addlinespace
 & \(n=99\) & \\*
\end{longtable*}

\begin{center}
\textbf{Lời giải}
\end{center}

\begin{enumerate}
\def\labelenumi{\alph{enumi}.}
\item
  Vì \(cf_2=40 <\frac{n}{2}=\frac{99}{2}= 49,5 < cf_3=60\), nên khẳng
  định nhóm 3 là nhóm đầu tiên có tần số tích lũy lớn hơn hoặc bằng 49,5
  là đúng.
\item
  Nhóm 3, \([37,5;42,5)\), có đầu mút trái \(r=37,5\), độ dài
  \(d=42,5-37,5=5\), tần số \(n_3=20\). Tần số tích lũy của nhóm 2 là
  \(cf_2=40\).
\item
  Giá trị \(M_e\) theo công thức là \begin{align*}
   M_e
       & = r + \left (\frac{49,5-cf_2}{n_3}\right)\cdot d \\
       & = 37,5 + \left(\frac{49,5-40}{20}\right)\cdot 5 \\
       & = 39,875 \text{ (g)}.
  \end{align*}
\end{enumerate}

\section*{Luyện tập 5}

Xác định trung vị của mẫu số liệu ghép nhóm ở \emph{Bảng 1}.

\begin{center}
    \textbf{Lời giải}
\end{center}

Thêm vào \emph{Bảng 1} cột tần số tích lũy để thu được bảng dưới đây.

\begin{longtable*}{ccc}
\toprule
Nhóm & Tần số & Tần số tích lũy\\
\midrule
\endfirsthead
\multicolumn{3}{@{}l}{\textit{(continued)}}\\
\toprule
Nhóm & Tần số & Tần số tích lũy\\
\midrule
\endhead

\endfoot
\bottomrule
\endlastfoot
\([0;4)\) & 13 & 13\\
\([4;8)\) & 29 & 42\\
\([8;12)\) & 48 & 90\\
\([12;16)\) & 22 & 112\\
\([16;20)\) & 8 & 120\\
\addlinespace
 & \(n=120\) & \\*
\end{longtable*}

Vì \[
    cf_2 = 42 < \frac{n}{2} = \frac{120}{2} = 60 < cf_3 = 90, 
\] nên nhóm 3, \([8;12)\), là nhóm đầu tiên có tần số tích lũy lớn hơn
hoặc bằng 60. Nóm này có đầu mút trái \(r=8\), độ dài \(d=4\), tần số
\(n_3=48\) và nhóm 2 có tần số tích lũy \(cf_2 = 42\).

Trung vị của mẫu số liệu là

\begin{align*}
    M_e
        & = r + \left ( \frac{\frac{n}{2} - cf_2}{n_3} \right )\cdot d \\
        & = 8 + \left (\frac{60 - 42}{48}\right)\cdot 4 \\
        & = \frac{19}{2} \\
        & = 9,5 \text{ (năm)}.
\end{align*}

\section*{Hoạt động 6}

Giáo viên chủ nhiệm chia thời gian sử dụng Internet trong một ngày của
40 học sinh thành năm nhóm (đơn vị: phút) và lập bảng tần số ghép nhóm
bao gồm cả tần số tích lũy như \emph{Bảng 12.}

\begin{enumerate}
\def\labelenumi{\alph{enumi}.}
\item
  Tìm trung vị \(M_e\) của mẫu số liệu ghép nhóm đó. Trung vị \(M_e\)
  còn gọi là tứ phân vị thứ hai \(Q_2\) của mẫu số liệu trên.
\item
  Nhóm 2 là nhóm đầu tiên có tần số tích lũy lớn hơn hoặc bằng \[
   \frac{n}{4} = \frac{40}{4} = 10 \text{ có đúng không?}
  \] Tìm đầu mút trái \(s\), độ dài \(h\), tần số \(n_2\) của nhóm 2 và
  tần số tích lũy \(cf_1\) của nhóm 1. Sau đó, hãy tính giá trị \(Q_1\)
  theo công thức sau: \[
   Q_1 = s + \left (\frac{10-cf_1}{n_2}\right)\cdot h.
  \] Giá trị nói trên được gọi là \emph{tứ phân vị thứ nhất} \(Q_1\) của
  mẫu số liệu đã cho.
\item
  Nhóm 3 là nhóm đầu tiên có tần số tích lũy lớn hơn hoặc bằng \[
   \frac{3n}{4} = \frac{3\cdot 40}{4} = 30 \text{ có đúng không?}
  \] Tìm đầu mút trái \(t\), độ dài \(l\), tần số \(n_3\) của nhóm 3 và
  tần số tích lũy \(cf_2\) của nhóm 2. Sau đó, hãy tính giá trị \(Q_3\)
  theo công thức sau: \[
   Q_3 = t + \left (\frac{30-cf_2}{n_3}\right)\cdot l.
  \] Giá trị nói trên được gọi là \emph{tứ phân vị thứ ba} \(Q_3\) của
  mẫu số liệu đã cho.
\end{enumerate}

\begin{longtable*}{ccc}
\toprule
Nhóm & Tần số & Tần số tích lũy\\
\midrule
\endfirsthead
\multicolumn{3}{@{}l}{\textit{(continued)}}\\
\toprule
Nhóm & Tần số & Tần số tích lũy\\
\midrule
\endhead

\endfoot
\bottomrule
\endlastfoot
\([0;60)\) & 6 & 6\\
\([60;120)\) & 13 & 19\\
\([120;180)\) & 13 & 32\\
\([180;240)\) & 6 & 38\\
\([240;300)\) & 2 & 40\\
\addlinespace
 & \(n=40\) & \\*
\end{longtable*}

\begin{center}
\textbf{Lời giải}
\end{center}

\begin{enumerate}
\def\labelenumi{\alph{enumi}.}
\tightlist
\item
  Quan sát từ cột Tần số tích lũy, \[
   cf_2 = 19 < \frac{n}{2} = \frac{40}{2} = 20 < cf_3 = 32,
  \] suy ra nhóm 3, \([120; 180)\), là nhóm đầu tiên có tần số tích lũy
  lớn hơn 20. Nhóm này có đầu mút trái \(r=120\), độ dài \(d=60\), tần
  số \(n_3=13\) và nhóm 2 có tần số tích lũy \(cf_2 = 19\). Trung vị
  \(M_e\) hay tứ phân vị thứ hai \(Q_2\) là
\end{enumerate}

\begin{align*}
    M_e = Q_2
        & = r + \left( \frac{\frac{n}{2} - cf_2}{n_3}\right)\cdot d \\
        & = 120 + \left( \frac{20-19}{13}\right)\cdot 60 \\
        & = \frac{1620}{13} \\
        & = 124,6154 \text{ (phút).}
\end{align*}

\begin{enumerate}
\def\labelenumi{\alph{enumi}.}
\setcounter{enumi}{1}
\tightlist
\item
  Quan sát từ cột Tần số tích lũy, \[
   cf_1 = 6 < \frac{n}{4} = \frac{40}{4} = 10 < cf_2 = 19,
  \] nên khẳng định nhóm 2 là nhóm đầu tiên có tần số tích lũy lớn hơn
  hoặc bằng 10 là khẳng định đúng. Nhóm 2, \([60; 120)\), có đầu mút
  trái \(s=60\), độ dài \(h=60\), tần số \(n_2=13\) và nhóm 1 có tần số
  tích lũy \(cf_1 = 6\). Tứ phân vị thứ nhất là
\end{enumerate}

\begin{align*}
    Q_1 
        & = s + \left (\frac{10-cf_1}{n_2}\right)\cdot h \\
        & = 60 + \left(\frac{10-6}{13}\right)\cdot 60 \\
        & = \frac{1020}{13} \\
        & = 78,4615 \text{ (phút).}
\end{align*}

\begin{enumerate}
\def\labelenumi{\alph{enumi}.}
\setcounter{enumi}{2}
\tightlist
\item
  Quan sát từ cột Tần số tích lũy, \[
   cf_2 = 19 < \frac{n}{4} = \frac{3\cdot 40}{4} = 30 < cf_3 = 32,
  \] nên khẳng định nhóm 3 là nhóm đầu tiên có tần số tích lũy lớn hơn
  hoặc bằng 30 là khẳng định đúng. Nhóm 3, \([120;180)\), có đầu mút
  trái \(t=120\), độ dài \(l=60\), tần số \(n_3=13\) và nhóm 2 có tần số
  tích lũy \(cf_2 = 19\). Tứ phân vị thứ ba là
\end{enumerate}

\begin{align*}
    Q_3 
        & = t + \left (\frac{30-cf_2}{n_3}\right)\cdot l \\
        & = 120 + \left(\frac{30-19}{13}\right)\cdot 60 \\
        & = \frac{2220}{13} \\
        & = 170,7692 \text{ (phút).}
\end{align*}

\section*{Luyện tập 6}

Tìm tứ phân vị của mẫu số liệu trong \emph{Bảng 1} (làm tròn các kết quả
đến hàng phần mười).

\begin{center}
\textbf{Lời giải}
\end{center}

\emph{Bảng 1} thêm vào cột Tần số tích lũy được trình bày như dưới đây.

\begin{longtable*}{ccc}
\toprule
Nhóm & Tần số & Tần số tích lũy\\
\midrule
\endfirsthead
\multicolumn{3}{@{}l}{\textit{(continued)}}\\
\toprule
Nhóm & Tần số & Tần số tích lũy\\
\midrule
\endhead

\endfoot
\bottomrule
\endlastfoot
\([0;4)\) & 13 & 13\\
\([4;8)\) & 29 & 42\\
\([8;12)\) & 48 & 90\\
\([12;16)\) & 22 & 112\\
\([16;20)\) & 8 & 120\\
\addlinespace
 & \(n=120\) & \\*
\end{longtable*}

\begin{itemize}
\tightlist
\item
  Quan sát từ cột Tần số tích lũy, \[
    cf_1 = 13 < \frac{n}{4} = \frac{120}{4} = 30 < cf_2 = 42,
  \] nên nhóm 2 là nhóm đầu tiên có tần số tích lũy lớn hơn 30. Nhóm 2,
  \([4; 8)\), có đầu mút trái \(s=4\), độ dài \(h=4\), tần số \(n_2=29\)
  và nhóm 1 có tần số tích lũy \(cf_1 = 13\). Tứ phân vị thứ nhất là
\end{itemize}

\begin{align*}
    Q_1 
        & = 4 + \left (\frac{30-13}{29}\right)\cdot 4 \\
        & = \frac{184}{29} \\
        & \approx 6 \text{ (năm).}
\end{align*}

\begin{itemize}
\tightlist
\item
  Quan sát từ cột Tần số tích lũy, \[
    cf_2 = 42 < \frac{n}{2} = \frac{120}{2} = 60 < cf_3 = 90,
  \] rút ra nhóm 3 là nhóm đầu tiên có tần số tích lũy lớn hơn 60. Nhóm
  3, \([8; 12)\), có đầu mút trái \(r=8\), độ dài \(d=4\), tần số
  \(n_3=48\) và nhóm 2 có tần số tích lũy \(cf_2 = 42\). Tứ phân vị thứ
  hai là
\end{itemize}

\begin{align*}
    Q_2
        & = 8 + \left( \frac{60-42}{48}\right)\cdot 4 \\
        & = \frac{19}{2} \\
        & \approx 10 \text{ (năm).}
\end{align*}

\begin{itemize}
\tightlist
\item
  Quan sát từ cột Tần số tích lũy, \[
    cf_2=42 < \frac{n}{4} = \frac{3\cdot 120}{4} = 90 = cf_3,
  \] nên nhóm 3, \([8;12)\), là nhóm đầu tiên có tần số tích lũy bằng
  90. Tứ phân vị thứ ba là
\end{itemize}

\begin{align*}
    Q_3
        & = 8 + \left(\frac{90-42}{48}\right)\cdot 4 \\
        & = 12 \text{ (năm).}
\end{align*}

\section*{Hoạt động 7}

Quan sát bảng tần số ghép nhóm bao gồm cả tần số tích luỹ ở \emph{Ví dụ
6} và cho biết:

\begin{enumerate}
\def\labelenumi{\alph{enumi}.}
\tightlist
\item
  Nhóm nào có tần số lớn nhất;
\item
  Đầu mút trái và độ dài của nhóm có tần số lớn nhất bằng bao nhiêu.
\end{enumerate}

\begin{center}
\textbf{Lời giải}
\end{center}

Bảng tần số ghép nhóm bao gồm cả tần số tích luỹ ở \emph{Ví dụ 6} được
trình bày như dưới đây.

\begin{longtable*}{ccc}
\toprule
Nhóm & Tần số & Tần số tích lũy\\
\midrule
\endfirsthead
\multicolumn{3}{@{}l}{\textit{(continued)}}\\
\toprule
Nhóm & Tần số & Tần số tích lũy\\
\midrule
\endhead

\endfoot
\bottomrule
\endlastfoot
\([30;40)\) & 2 & 2\\
\([40;50)\) & 10 & 12\\
\([50;60)\) & 16 & 28\\
\([60;70)\) & 8 & 36\\
\([70;80)\) & 2 & 38\\
\addlinespace
\([80;90)\) & 2 & 40\\
 & \(n=40\) & \\*
\end{longtable*}

\begin{enumerate}
\def\labelenumi{\alph{enumi}.}
\tightlist
\item
  Nhóm 3, \([50; 60)\), có tần số lớn nhất 16.
\item
  Nhóm 3 có dầu mút trái 50 và độ dài 10.
\end{enumerate}

\section*{Luyện tập 7}

Tìm mốt của mẫu số liệu trong \emph{Ví dụ 6} (làm tròn các kết quả đến
hàng phần mười).

\begin{center}
\textbf{Lời giải}
\end{center}

Bảng tần số ghép nhóm của mẫu số liệu \emph{Ví dụ 6} được trình bày dưới
đây.

\begin{longtable*}{cc}
\toprule
Nhóm & Tần số\\
\midrule
\endfirsthead
\multicolumn{2}{@{}l}{\textit{(continued)}}\\
\toprule
Nhóm & Tần số\\
\midrule
\endhead

\endfoot
\bottomrule
\endlastfoot
\([30;40)\) & 2\\
\([40;50)\) & 10\\
\([50;60)\) & 16\\
\([60;70)\) & 8\\
\([70;80)\) & 2\\
\addlinespace
\([80;90)\) & 2\\
 & \(n=40\)\\*
\end{longtable*}

Quan sát cột Tần số, nhóm 3, \([50;60)\), là nhóm có tần số lớn nhất
\(n_3=16\), có đầu mút trái \(u=50\) và độ dài \(g=10\). Nhóm 2 có tần
số \(n_2=10\) và nhóm 4 có tần số \(n_4=8\). Mốt là

\begin{align*}
    M_o
        & = u + \left(\frac{n_3-n_2}{2n_3-n_2-n_4}\right)\cdot g \\
        & = 50 + \left(\frac{16-10}{2\cdot 16-10-8}\right)\cdot 10 \\
        & = \frac{380}{7} \\
        & = 54,3 \text{ (kg).}
\end{align*}

\section{Bài tập}\label{buxe0i-tux1eadp}

Xem Đáp án để tra nhanh kết quả.

\section*{Bài tập 1}

Mẫu số liệu dưới đây ghi lại tốc độ của 40 ô-tô khi đi qua một trạm đo
tốc độ (đơn vị: km/h):

\begin{table}[!h]
\centering
\begin{tabular}{cccccccccc}
\toprule
48,5 & 43,0 & 50,0 & 55,0 & 45,0 & 60,0 & 53,0 & 55,5 & 44,0 & 65,0\\
51,0 & 62,5 & 41,0 & 44,5 & 57,0 & 57,0 & 68,0 & 49,0 & 46,5 & 53,5\\
61,0 & 49,5 & 54,0 & 62,0 & 59,0 & 56,0 & 47,0 & 50,0 & 60,0 & 61,0\\
49,5 & 52,5 & 57,0 & 47,0 & 60,0 & 55,0 & 45,0 & 47,5 & 48,0 & 61,5\\
\bottomrule
\end{tabular}
\end{table}

\begin{enumerate}
\def\labelenumi{\alph{enumi}.}
\tightlist
\item
  Lập bảng tần số ghép nhóm cho mẫu số liệu trên có sáu nhóm ứng với sáu
  nửa khoảng:
\end{enumerate}

\begin{center}
$[40; 45)$, $[45 ; 50)$, $[50; 55)$, $[55 ; 60)$, $[60; 65)$, $[65; 70)$.
\end{center}

\begin{enumerate}
\def\labelenumi{\alph{enumi}.}
\setcounter{enumi}{1}
\item
  Xác định trung bình cộng, trung vị, tứ phân vị của mẫu số liệu ghép
  nhóm trên.
\item
  Mốt của mẫu số liệu ghép nhóm trên là bao nhiêu?
\end{enumerate}

\begin{center}
\textbf{Lời giải}
\end{center}

\begin{enumerate}
\def\labelenumi{\alph{enumi}.}
\tightlist
\item
  Để đảm bảo phân nhóm chính xác, số liệu nên được sắp xếp theo thứ tự
  tăng dần.
\end{enumerate}

\begin{longtable*}{cccccccccc}
\toprule
\endfirsthead
\multicolumn{10}{@{}l}{\textit{(continued)}}\\
\toprule
\endhead

\endfoot
\bottomrule
\endlastfoot
41,0 & 43,0 & 44,0 & 44,5 & 45,0 & 45,0 & 46,5 & 47,0 & 47,0 & 47,5\\
48,0 & 48,5 & 49,0 & 49,5 & 49,5 & 50,0 & 50,0 & 51,0 & 52,5 & 53,0\\
53,5 & 54,0 & 55,0 & 55,0 & 55,5 & 56,0 & 57,0 & 57,0 & 57,0 & 59,0\\
60,0 & 60,0 & 60,0 & 61,0 & 61,0 & 61,5 & 62,0 & 62,5 & 65,0 & 68,0\\*
\end{longtable*}

Nhìn vào dãy có thứ tự tăng dần, dễ dàng thấy rằng thuộc vào nhóm
\([40;45)\) có bốn số liệu bao gồm 41; 43; 44; 44,5. Bảng tần số ghép
nhóm được thành lập dựa trên các quan sát như thế cho đến khi duyệt hết
số liệu.

\begin{longtable*}{cc}
\toprule
Nhóm & Tần số\\
\midrule
\endfirsthead
\multicolumn{2}{@{}l}{\textit{(continued)}}\\
\toprule
Nhóm & Tần số\\
\midrule
\endhead

\endfoot
\bottomrule
\endlastfoot
\([40;45)\) & 4\\
\([45;50)\) & 11\\
\([50;55)\) & 7\\
\([55;60)\) & 8\\
\([60;65)\) & 8\\
\addlinespace
\([65;70)\) & 2\\
 & \(n=40\)\\*
\end{longtable*}

\begin{enumerate}
\def\labelenumi{\alph{enumi}.}
\setcounter{enumi}{1}
\tightlist
\item
  Thêm vào bảng tần số ghép nhóm ở trên cột số thứ tự để dễ quan sát,
  cột Giá trị đại diện để tính trung bình, và cột Tần số tích lũy để
  tính tứ phân vị.
\end{enumerate}

\begin{longtable*}{ccccc}
\toprule
  & Nhóm & Giá trị đại diện & Tần số & Tần số tích lũy\\
\midrule
\endfirsthead
\multicolumn{5}{@{}l}{\textit{(continued)}}\\
\toprule
  & Nhóm & Giá trị đại diện & Tần số & Tần số tích lũy\\
\midrule
\endhead

\endfoot
\bottomrule
\endlastfoot
1 & \([40;45)\) & 42,5 & 4 & 4\\
2 & \([45;50)\) & 47,5 & 11 & 15\\
3 & \([50;55)\) & 52,5 & 7 & 22\\
4 & \([55;60)\) & 57,5 & 8 & 30\\
5 & \([60;65)\) & 62,5 & 8 & 38\\
\addlinespace
6 & \([65;70)\) & 67,5 & 2 & 40\\
 &  &  & \(n=40\) & \\*
\end{longtable*}

\begin{itemize}
\tightlist
\item
  Mẫu có \(n=40\) số liệu. Nhóm 1, \([40;45)\), có trung điểm
  \(x_1=42,5\) làm giá trị đại diện và có tần số \(n_1=4\). Ký hiệu
  tương tự cho các nhóm còn lại. Trung bình cộng là \begin{align*}
    \overline{x} 
        & = \frac{1}{n} (n_1\cdot x_1 + n_2\cdot x_2 + n_3\cdot x_3 + n_4\cdot x_4 + n_5\cdot x_5 + n_6\cdot x_6) \\
        & = \frac{1}{40} (4\cdot 42,5 + 11\cdot 47,5 + 7\cdot 52,5 + 8\cdot 57,5 + 8\cdot 62,5 + 2\cdot 67,5) \\
        & = \frac{431}{8} \\
        & = 53,875 \text{ (km/h)}.
  \end{align*}
\end{itemize}

\begin{itemize}
\tightlist
\item
  Quan sát cột Tần số tích lũy, ta thấy \[
    cf_2=15 < \frac{n}{2}=\frac{40}{2}=20 < cf_3=22.
  \] Suy ra nhóm 3, \([50;55)\), là nhóm đầu tiên có tần số tích lũy
  không nhỏ hơn 20. Nhóm này có đầu mút trái \(r=50\), độ dài \(d=5\) và
  tần số \(n_3=7\). Trung vị là \begin{align*}
        M_e
            & = r+\left(\frac{\frac{n}{2}-cf_2}{n_3}\right)\cdot d \\
            & = 50 + \left(\frac{20-15}{7}\right)\cdot 5 \\
            & = \frac{375}{7} \\
            & \approx 53,5714 \text{ (km/h).}
    \end{align*}
\end{itemize}

\begin{itemize}
\item
  Quan sát cột Tần số tích lũy, ta thấy \[
  cf_1=4 < \frac{n}{4}=\frac{40}{4}=10 < cf_2=15.
  \] Suy ra nhóm 2, \([45;50)\), là nhóm đầu tiên có tần số tích lũy
  không nhỏ hơn 10. Nhóm này có đầu mút trái \(s=45\), độ dài \(h=5\) và
  tần số \(n_2=11\). Tứ phân vị thứ nhất là \begin{align*}
        Q_1
            & = s + \left( \frac{\frac{n}{4}-cf_1}{n_2}\right)\cdot h \\
            & = 45 + \left(\frac{10-4}{11}\right)\cdot 5 \\
            & = \frac{525}{11} \\
            & \approx 47,7273 \text{ (km/h).}
    \end{align*}
\item
  Tứ phân vị thứ hai \(Q_2\) chính là trung vị \(M_e\), hay
  \(Q_2 = M_e = 53,5714\) (km/h).
\end{itemize}

\begin{itemize}
\tightlist
\item
  Quan sát cột Tần số tích lũy, ta thấy \[
  cf_3=22 < \frac{3n}{4} = \frac{3 \cdot 40}{4} = 30 = cf_4.
  \] Suy ra nhóm 4, \([55;60)\), là nhóm đầu tiên có tần số tích lũy
  bằng với 30. Nhóm này có đầu mút trái \(t=55\), độ dài \(l=5\) và tần
  số \(n_4=8\). Tứ phân vị thứ ba là \begin{align*}
        Q_3
            & = t + \left(\frac{\frac{3n}{4}-cf_3}{n_4}\right)\cdot l \\
            & = 55 + \left(\frac{30-22}{8}\right)\cdot 5 \\
            & = 60 \text{ (km/h).}
    \end{align*}
\end{itemize}

\begin{enumerate}
\def\labelenumi{\alph{enumi}.}
\setcounter{enumi}{2}
\tightlist
\item
  Quan sát cột Tần số, nhóm 2, \([45;50)\), là nhóm có tần số lớn nhất
  \(n_2=11\). Nó có đầu mút trái \(u=45\) và độ dài \(g=5\). Nhóm 1 có
  tần số \(n_1=4\) và nhóm 3 có tần số \(n_3=7\). Mốt là \begin{align*}
       M_o
           & = u + \left(\frac{n_2-n_1}{2n_2-n_1-n_3}\right) \cdot g \\
           & = 45 + \left(\frac{11-4}{2\cdot 11-4-7}\right)\cdot 5 \\
           & = \frac{530}{11} \\
           & \approx 48,1818 \text{ (km/h)}.
   \end{align*}
\end{enumerate}

Tóm lại, bảng tần số ghép nhóm là

\begin{center}
\centering
\begin{tabular}{|c|c|c|c|c|c|}
\hline 
$[40; 45)$ & $[45 ; 50)$ & $[50; 55)$ & $[55 ; 60)$ & $[60; 65)$ & $[65; 70)$ \\
\hline 
4 & 11 & 7 & 8 & 8 & 2. \\
\hline
\end{tabular}
\end{center}

Các số đặc trưng đo xu thế trung tâm là

\begin{center}
\begin{tabular}{|c|c|c|c|c|}
\hline
$\overline{x}$ & $Q_1$ & $Q_2$ ($M_e$) & $Q_3$ & $M_o$ \\
\hline
53,875 & 47,7273 & 53,5714 & 60 & 48,1818. \\
\hline 
\end{tabular}
\end{center}

\section*{Bài tập 2}

Mẫu số liệu sau ghi lại cân nặng của 30 bạn học sinh (đơn vị: kg):

\begin{longtable*}{cccccccccc}
\toprule
\endfirsthead
\multicolumn{10}{@{}l}{\textit{(continued)}}\\
\toprule
\endhead

\endfoot
\bottomrule
\endlastfoot
17,0 & 40,0 & 39,0 & 40,5 & 42,0 & 51,0 & 41,5 & 39,0 & 41,0 & 30,0\\
40,0 & 42,0 & 40,5 & 39,5 & 41,0 & 40,5 & 37,0 & 39,5 & 40,0 & 41,0\\
38,5 & 39,5 & 40,0 & 41,0 & 39,0 & 40,5 & 40,0 & 38,5 & 39,5 & 41,5\\*
\end{longtable*}

\begin{enumerate}
\def\labelenumi{\alph{enumi}.}
\tightlist
\item
  Lập bảng tần số ghép nhóm cho mẫu số liệu trên có tám nhóm ứng với tám
  nửa khoảng:
\end{enumerate}

\begin{center}
$[15; 20)$, $[20; 25)$, $[25; 30)$, $[30; 35)$, $[35; 40)$, $[40 ; 45)$, $[45; 50)$, $[50; 55)$.
\end{center}

\begin{enumerate}
\def\labelenumi{\alph{enumi}.}
\setcounter{enumi}{1}
\item
  Xác định trung bình cộng, trung vị, tứ phân vị của mẫu số liệu ghép
  nhóm trên.
\item
  Mốt của mẫu số liệu ghép nhóm trên là bao nhiêu?
\end{enumerate}

\begin{center}
\textbf{Lời giải}
\end{center}

\begin{enumerate}
\def\labelenumi{\alph{enumi}.}
\tightlist
\item
  Để đảm bảo phân nhóm chính xác, số liệu nên được sắp xếp theo thứ tự
  tăng dần.
\end{enumerate}

\begin{longtable*}{cccccccccc}
\toprule
\endfirsthead
\multicolumn{10}{@{}l}{\textit{(continued)}}\\
\toprule
\endhead

\endfoot
\bottomrule
\endlastfoot
17,0 & 30,0 & 37,0 & 38,5 & 38,5 & 39,0 & 39,0 & 39,0 & 39,5 & 39,5\\
39,5 & 39,5 & 40,0 & 40,0 & 40,0 & 40,0 & 40,0 & 40,5 & 40,5 & 40,5\\
40,5 & 41,0 & 41,0 & 41,0 & 41,0 & 41,5 & 41,5 & 42,0 & 42,0 & 51,0\\*
\end{longtable*}

Nhìn vào dãy có thứ tự tăng dần, dễ dàng thấy rằng thuộc vào nhóm
\([15,20)\) chỉ có một số liệu là 17. Bảng tần số ghép nhóm được thành
lập dựa trên các quan sát như thế cho đến khi duyệt hết số liệu.

\begin{longtable*}{cc}
\toprule
Nhóm & Tần số\\
\midrule
\endfirsthead
\multicolumn{2}{@{}l}{\textit{(continued)}}\\
\toprule
Nhóm & Tần số\\
\midrule
\endhead

\endfoot
\bottomrule
\endlastfoot
\([15;20)\) & 1\\
\([20;25)\) & 0\\
\([25;30)\) & 0\\
\([30;35)\) & 1\\
\([35;40)\) & 10\\
\addlinespace
\([40;45)\) & 17\\
\([45;50)\) & 0\\
\([50;55)\) & 1\\
 & \(n=30\)\\*
\end{longtable*}

\begin{enumerate}
\def\labelenumi{\alph{enumi}.}
\setcounter{enumi}{1}
\tightlist
\item
  Thêm vào bảng tần số ghép nhóm ở trên cột số thứ tự để dễ quan sát,
  cột Giá trị đại diện để tính trung bình, và cột Tần số tích lũy để
  tính tứ phân vị.
\end{enumerate}

\begin{longtable*}{ccccc}
\toprule
  & Nhóm & Giá trị đại diện & Tần số & Tần số tích lũy\\
\midrule
\endfirsthead
\multicolumn{5}{@{}l}{\textit{(continued)}}\\
\toprule
  & Nhóm & Giá trị đại diện & Tần số & Tần số tích lũy\\
\midrule
\endhead

\endfoot
\bottomrule
\endlastfoot
1 & \([15;20)\) & 17,5 & 1 & 1\\
2 & \([20;25)\) & 22,5 & 0 & 1\\
3 & \([25;30)\) & 27,5 & 0 & 1\\
4 & \([30;35)\) & 32,5 & 1 & 2\\
5 & \([35;40)\) & 37,5 & 10 & 12\\
\addlinespace
6 & \([40;45)\) & 42,5 & 17 & 29\\
7 & \([45;50)\) & 47,5 & 0 & 29\\
8 & \([50;55)\) & 52,5 & 1 & 30\\
 &  &  & \(n=30\) & \\*
\end{longtable*}

\begin{itemize}
\tightlist
\item
  Mẫu có \(n=30\) số liệu. Nhóm 1, \([15;20)\), có trung điểm
  \(x_1=17,5\) làm giá trị đại diện và có tần số \(n_1=1\). Ký hiệu
  tương tự với các nhóm còn lại. Trung bình là \begin{align*}
  \overline{x}
    & = \frac{1}{n}(n_1\cdot x_1 + n_2\cdot x_2 + n_3\cdot x_3 + n_4\cdot x_4 + n_5\cdot x_5 + n_6\cdot x_6 + n_7\cdot x_7 + n_8\cdot x_8) \\
    & = \frac{1}{30} (1\cdot 17,5 + 0\cdot 22,5 + 0\cdot 27,5+ 1\cdot 32,5 + 10\cdot 37,5 + 17\cdot 42,5 + 0\cdot 47,5 + 52,5 \cdot 1) \\
    & = 40 \text{ (kg).}
  \end{align*}
\end{itemize}

\begin{itemize}
\tightlist
\item
  Quan sát cột Tần số tích lũy, ta thấy \[
  cf_5=12 < \frac{n}{2}=15 < cf_6=29.
  \] Suy ra nhóm 6, \([40;45)\), là nhóm đầu tiên có tần số tích lũy
  không nhỏ hơn 15. Nhóm này có đầu mút trái \(r=40\), độ dài \(d=5\) và
  tần số \(n_6=17\). Trung vị là \begin{align*}
  M_e
    & = r + \left(\frac{\frac{n}{2}-cf_5}{n_6}\right)\cdot d \\
    & = 40 + \left(\frac{15-12}{17}\right)\cdot 5 \\
    & = \frac{695}{17} \\
    & \approx 40,8824 \text{ (kg).}
  \end{align*}
\end{itemize}

\begin{itemize}
\item
  Quan sát cột Tần số tích lũy, ta thấy \[
  cf_4=2 <\frac{n}{4}=\frac{30}{4}=7,5 < cf_5=12.
  \] Suy ra nhóm 5, \([35;40)\), là nhóm đầu tiên có tần số tích lũy,
  không nhỏ hơn 7,5. Nhóm này có đầ mút trái \(s=35\), độ dài \(h=5\) và
  tần số \(n_5=10\). Tứ phân vị thứ nhất là \begin{align*}
        Q_1
            & = s + \left( \frac{\frac{n}{4}-cf_4}{n_5}\right)\cdot h \\
            & = 35 + \left(\frac{7,5-2}{10}\right)\cdot 5 \\
            & = \frac{151}{4} \\
            & = 37,75 \text{ (kg).}
    \end{align*}
\item
  Tứ phân vị thứ hai \(Q_2\) chính là trung vị \(M_e\), hay
  \(Q_2=M_e \approx 40,8824\) (kg).
\end{itemize}

\begin{itemize}
\tightlist
\item
  Quan sát cột Tần số tích lũy, ta có \[
  cf_5=12<\frac{3n}{4}=\frac{3\cdot 30}{3}=22,5<cf_6=29.
  \] Suy ra nhóm 6, \([40;45)\) là nhòm đầu tiên có tần số tích lũy
  không nhỏ hơn 22,5. Nhóm này có đầu mút trái \(t=40\), độ dài \(l=5\)
  và tần số \(n_6=17\). Tứ phân vị thứ ba là \begin{align*}
        Q_3
            & = t + \left(\frac{\frac{3n}{4}-cf_5}{n_6}\right)\cdot l \\
            & = 40 +\left(\frac{22,5-12}{17}\right) \cdot 5 \\
            & = \frac{1465}{34} \\
            & \approx 43,0882 \text{ (kg).} 
    \end{align*}
\end{itemize}

\begin{enumerate}
\def\labelenumi{\alph{enumi}.}
\setcounter{enumi}{2}
\tightlist
\item
  Quan sát cột Tần số, nhóm 6, \([40,45)\) là nhóm có tần số lớn nhất
  \(n_6=17\). Nhóm này có đầu mút trái \(u=40\) và độ dài \(g=5\). Nhóm
  5 liền trước nó có tần số \(n_5=10\) và nhóm 7 liền sau nó có tần số
  \(n_7=0\). Mốt là \begin{align*}
       M_o 
           & = u + \left(\frac{n_6-n_5}{2n_6 - n_5 - n_7}\right)\cdot g \\
           & = 40 + \left(\frac{17-10}{2\cdot 17 - 10 - 0}\right)\cdot 5 \\
           & = \frac{995}{24} \\
           & \approx 41,5 \text{ (kg).}
   \end{align*}
\end{enumerate}

Tóm lại, bảng tần số ghép nhóm là

\begin{center}
\begin{tabular}{|c|c|c|c|c|c|c|c|}
\hline 
$[15;20)$ & $[20; 25)$ & $[25; 30)$ & $[30; 35)$ & $[35;40)$ & $[40; 45)$ & $[45;50)$ & $[50;55)$ \\
\hline 
1 & 0 & 0 & 1 & 10 & 17 & 0 & 1. \\
\hline
\end{tabular}
\end{center}

Các số đặc trưng đo xu thế trung tâm là

\begin{center}
\begin{tabular}{|c|c|c|c|c|}
\hline
$\overline{x}$ & $Q_1$ & $Q_2$ ($M_e$) & $Q_3$ & $M_o$ \\
\hline
40 & 37,75 & 40,8824 & 43,0882 & 41,4583. \\
\hline 
\end{tabular}
\end{center}

\section*{Bài tập 3}

Bảng 15 cho ta bảng tần số ghép nhóm số liệu thống kê chiều cao của 40
mẫu cây ở một vườn thực vật (đơn vị: cm).

\begin{enumerate}
\def\labelenumi{\alph{enumi}.}
\item
  Xác định trung bình cộng, trung vị, tứ phân vị của mẫu số liệu ghép
  nhóm trên.
\item
  Mốt của mẫu số liệu ghép nhóm trên là bao nhiêu?
\end{enumerate}

\begin{longtable*}{ccc}
\toprule
Nhóm & Tần số & Tần số tích lũy\\
\midrule
\endfirsthead
\multicolumn{3}{@{}l}{\textit{(continued)}}\\
\toprule
Nhóm & Tần số & Tần số tích lũy\\
\midrule
\endhead

\endfoot
\bottomrule
\endlastfoot
\([30;40)\) & 4 & 4\\
\([40;50)\) & 10 & 14\\
\([50;60)\) & 14 & 28\\
\([60;70)\) & 6 & 34\\
\([70;80)\) & 4 & 38\\
\addlinespace
\([80;90)\) & 2 & 40\\
 & \(n=40\) & \\*
\end{longtable*}

\begin{center}
\textbf{Lời giải}
\end{center}

a.Thêm vào bảng đã cho cột Giá trị đại diện để tính trung bình, và cột
số thứ tự để dễ quan sát.

\begin{longtable*}{ccccc}
\toprule
  & Nhóm & Giá trị đại diện & Tần số & Tần số tích lũy\\
\midrule
\endfirsthead
\multicolumn{5}{@{}l}{\textit{(continued)}}\\
\toprule
  & Nhóm & Giá trị đại diện & Tần số & Tần số tích lũy\\
\midrule
\endhead

\endfoot
\bottomrule
\endlastfoot
1 & \([30;40)\) & 35 & 4 & 4\\
2 & \([40;50)\) & 45 & 10 & 14\\
3 & \([50;60)\) & 55 & 14 & 28\\
4 & \([60;70)\) & 65 & 6 & 34\\
5 & \([70;80)\) & 75 & 4 & 38\\
\addlinespace
6 & \([80;90)\) & 85 & 2 & 40\\
 &  &  & \(n=40\) & \\*
\end{longtable*}

\begin{itemize}
\tightlist
\item
  Mẫu có \(n=40\) số liệu. Nhóm 1 có giá trị đại diện \(x_1=35\) và tần
  số \(n_1=4\), vân vân cho đến nhóm 6 có giá trị đại diện \(x_6=85\) và
  tần số \(n_6=2\). Trung bình là
\end{itemize}

\begin{align*}
    \overline{x}
        & = \frac{1}{n}(n_1\cdot x_1 + n_2\cdot x_2 + n_3\cdot x_3 + n_4\cdot x_4 + n_5\cdot x_5 + n_6\cdot x_6) \\
        & = \frac{1}{40} (4\cdot 35 + 10\cdot 45 + 14\cdot 55 + 6\cdot 65 + 4\cdot 75 + 2\cdot 85) \\
        & = \frac{111}{2} \\
        & = 55,5 \text{ (cm).}
\end{align*}

\begin{itemize}
\item
  Quan sát cột Tần số tích lũy, ta thấy \[
  cf_2=14 <\frac{n}{2}=\frac{40}{2}=20 <cf_3=28.
  \] Suy ra nhóm 3, \([50;55)\), là nhóm đầu tiên có tần số tích lũy
  không nhỏ hơn 20. Nhóm này có đầu mút trái \(r=50\), độ dài \(d=10\)
  và tần số \(n_3=14\). Trung vị là

  \begin{align*}
        M_e
            & = r + \left(\frac{\frac{n}{2}-cf_2}{n_3}\right)\cdot d \\
            & = 50 + \left(\frac{20-14}{14}\right)\cdot 10 \\
            & = \frac{380}{7} \\
            & \approx 54,2857 \text{ (cm).}
    \end{align*}
\end{itemize}

\begin{itemize}
\item
  Quan sát cột Tần số tích lũy, ta thấy \[
  cf_1=4 < \frac{n}{4}=\frac{40}{4}=10 < cf_2=14.
  \] Suy ra nhóm 2, \([40;50)\), là nhóm đầu tiên có tần số tích lũy
  không nhỏ hơn 10. Nhóm này có đầu mút trái \(s=40\), độ dài \(h=10\)
  và tần số \(n_2=10\). Tứ phân vị thứ nhất là

  \begin{align*}
        Q_1
            & = s + \left( \frac{\frac{n}{4}-cf_1}{n_2}\right)\cdot h \\
            & = 40 + \left(\frac{10-4}{10}\right)\cdot 10 \\
            & = 46 \text{ (cm).}
    \end{align*}
\item
  Tứ phân vị thứ hai \(Q_2\) chính là trung vị \(M_e\), hay
  \(Q_2 = M_e = 54,2857\) (cm).
\end{itemize}

\begin{itemize}
\item
  Quan sát cột Tần số tích lũy, ta thấy \[
  cf_3=28<\frac{3n}{4}=\frac{3\cdot 40}{4}=30<cf_4=34.
  \] Suy ra nhóm 4, \([60;70)\), là nhóm đầu tiên có tần số tích lũy
  không nhỏ hơn 30. Nhóm này có đầu mút trái \(t=60\), độ dài \(l=10\)
  và tần số \(n_4=6\). Tứ phân vị thứ ba là

  \begin{align*}
        Q_3
            & = t + \left(\frac{\frac{3n}{4}-cf_3}{n_4}\right)\cdot l \\
            & = 60 + \left(\frac{30-28}{6}\right)\cdot 10 \\
            & = \frac{190}{3} \\
            & = 63,3333 \text{ (cm).}
    \end{align*}
\end{itemize}

\begin{enumerate}
\def\labelenumi{\alph{enumi}.}
\setcounter{enumi}{1}
\tightlist
\item
  Quan sát cột Tần số, nhóm 3, \([50;60)\), là nhóm có tần số lớn nhất
  \(n_3=14\). Nó có đầu mút trái \(u=50\) và độ dài \(g=10\). Nhóm 2 có
  tần số \(n_2=10\) và nhóm 4 có tần số \(n_4=6\). Mốt là \begin{align*}
       M_o
           & = u + \left(\frac{n_3-n_2}{2n_3-n_2-n_4}\right) \cdot g \\
           & = 50 + \left(\frac{14-10}{2\cdot 14-10-6}\right)\cdot 10 \\
           & = \frac{160}{3} \\
           & \approx 53,3333 \text{ (cm)}.
   \end{align*}
\end{enumerate}

Tóm lại, bảng tần số ghép nhóm là

\begin{center}
\begin{tabular}{|c|c|c|c|c|c|}
\hline 
$[30;40)$ & $[40;50)$ & $[50;60)$ & $[60;70)$ & $[70;80)$ & $[80;90)$  \\
\hline 
1 & 0 & 0 & 1 & 10 & 17. \\
\hline
\end{tabular}
\end{center}

Các số đặc trưng đo xu thế trung tâm là

\begin{center}
\begin{tabular}{|c|c|c|c|c|}
\hline
$\overline{x}$ & $Q_1$ & $Q_2$ ($M_e$) & $Q_3$ & $M_o$ \\
\hline
55,5 & 46 & 54,2857 & 63,3333 & 53,3333. \\
\hline 
\end{tabular}
\end{center}




\end{document}
