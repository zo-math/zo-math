% Options for packages loaded elsewhere
\PassOptionsToPackage{unicode}{hyperref}
\PassOptionsToPackage{hyphens}{url}
\PassOptionsToPackage{dvipsnames,svgnames,x11names}{xcolor}
%
\documentclass[
  letterpaper,
  DIV=11,
  numbers=noendperiod]{scrartcl}

\usepackage{amsmath,amssymb}
\usepackage{iftex}
\ifPDFTeX
  \usepackage[T1]{fontenc}
  \usepackage[utf8]{inputenc}
  \usepackage{textcomp} % provide euro and other symbols
\else % if luatex or xetex
  \usepackage{unicode-math}
  \defaultfontfeatures{Scale=MatchLowercase}
  \defaultfontfeatures[\rmfamily]{Ligatures=TeX,Scale=1}
\fi
\usepackage{lmodern}
\ifPDFTeX\else  
    % xetex/luatex font selection
\fi
% Use upquote if available, for straight quotes in verbatim environments
\IfFileExists{upquote.sty}{\usepackage{upquote}}{}
\IfFileExists{microtype.sty}{% use microtype if available
  \usepackage[]{microtype}
  \UseMicrotypeSet[protrusion]{basicmath} % disable protrusion for tt fonts
}{}
\makeatletter
\@ifundefined{KOMAClassName}{% if non-KOMA class
  \IfFileExists{parskip.sty}{%
    \usepackage{parskip}
  }{% else
    \setlength{\parindent}{0pt}
    \setlength{\parskip}{6pt plus 2pt minus 1pt}}
}{% if KOMA class
  \KOMAoptions{parskip=half}}
\makeatother
\usepackage{xcolor}
\setlength{\emergencystretch}{3em} % prevent overfull lines
\setcounter{secnumdepth}{-\maxdimen} % remove section numbering
% Make \paragraph and \subparagraph free-standing
\makeatletter
\ifx\paragraph\undefined\else
  \let\oldparagraph\paragraph
  \renewcommand{\paragraph}{
    \@ifstar
      \xxxParagraphStar
      \xxxParagraphNoStar
  }
  \newcommand{\xxxParagraphStar}[1]{\oldparagraph*{#1}\mbox{}}
  \newcommand{\xxxParagraphNoStar}[1]{\oldparagraph{#1}\mbox{}}
\fi
\ifx\subparagraph\undefined\else
  \let\oldsubparagraph\subparagraph
  \renewcommand{\subparagraph}{
    \@ifstar
      \xxxSubParagraphStar
      \xxxSubParagraphNoStar
  }
  \newcommand{\xxxSubParagraphStar}[1]{\oldsubparagraph*{#1}\mbox{}}
  \newcommand{\xxxSubParagraphNoStar}[1]{\oldsubparagraph{#1}\mbox{}}
\fi
\makeatother


\providecommand{\tightlist}{%
  \setlength{\itemsep}{0pt}\setlength{\parskip}{0pt}}\usepackage{longtable,booktabs,array}
\usepackage{calc} % for calculating minipage widths
% Correct order of tables after \paragraph or \subparagraph
\usepackage{etoolbox}
\makeatletter
\patchcmd\longtable{\par}{\if@noskipsec\mbox{}\fi\par}{}{}
\makeatother
% Allow footnotes in longtable head/foot
\IfFileExists{footnotehyper.sty}{\usepackage{footnotehyper}}{\usepackage{footnote}}
\makesavenoteenv{longtable}
\usepackage{graphicx}
\makeatletter
\newsavebox\pandoc@box
\newcommand*\pandocbounded[1]{% scales image to fit in text height/width
  \sbox\pandoc@box{#1}%
  \Gscale@div\@tempa{\textheight}{\dimexpr\ht\pandoc@box+\dp\pandoc@box\relax}%
  \Gscale@div\@tempb{\linewidth}{\wd\pandoc@box}%
  \ifdim\@tempb\p@<\@tempa\p@\let\@tempa\@tempb\fi% select the smaller of both
  \ifdim\@tempa\p@<\p@\scalebox{\@tempa}{\usebox\pandoc@box}%
  \else\usebox{\pandoc@box}%
  \fi%
}
% Set default figure placement to htbp
\def\fps@figure{htbp}
\makeatother

% Cấu hình font chữ chính và toán học
\usepackage{fontspec}
\usepackage{unicode-math}
\setmainfont{STIX Two Text}
\setsansfont{STIX Two Text}
\setmonofont{STIX Two Text}
\setmathfont{STIX Two Math}

% Cấu hình kích thước chữ
% \fontsize{12pt}{14pt}\selectfont

% Đánh số section
\setcounter{secnumdepth}{3}

% Các gói hỗ trợ bổ sung
\usepackage{hyperref}  % Hỗ trợ liên kết
\usepackage{graphicx}  % Hỗ trợ hình ảnh
\usepackage{amsmath, amssymb}  % Hỗ trợ toán học nâng cao
\usepackage{booktabs}
\usepackage{longtable}
\usepackage{array}
\usepackage{multirow}
\usepackage{wrapfig}
\usepackage{float}
\usepackage{colortbl}
\usepackage{pdflscape}
\usepackage{tabu}
\usepackage{threeparttable}
\usepackage{threeparttablex}
\usepackage[normalem]{ulem}
\usepackage{makecell}
\usepackage{xcolor}
\KOMAoption{captions}{tableheading}
\makeatletter
\@ifpackageloaded{caption}{}{\usepackage{caption}}
\AtBeginDocument{%
\ifdefined\contentsname
  \renewcommand*\contentsname{Table of contents}
\else
  \newcommand\contentsname{Table of contents}
\fi
\ifdefined\listfigurename
  \renewcommand*\listfigurename{List of Figures}
\else
  \newcommand\listfigurename{List of Figures}
\fi
\ifdefined\listtablename
  \renewcommand*\listtablename{List of Tables}
\else
  \newcommand\listtablename{List of Tables}
\fi
\ifdefined\figurename
  \renewcommand*\figurename{Figure}
\else
  \newcommand\figurename{Figure}
\fi
\ifdefined\tablename
  \renewcommand*\tablename{Table}
\else
  \newcommand\tablename{Table}
\fi
}
\@ifpackageloaded{float}{}{\usepackage{float}}
\floatstyle{ruled}
\@ifundefined{c@chapter}{\newfloat{codelisting}{h}{lop}}{\newfloat{codelisting}{h}{lop}[chapter]}
\floatname{codelisting}{Listing}
\newcommand*\listoflistings{\listof{codelisting}{List of Listings}}
\makeatother
\makeatletter
\makeatother
\makeatletter
\@ifpackageloaded{caption}{}{\usepackage{caption}}
\@ifpackageloaded{subcaption}{}{\usepackage{subcaption}}
\makeatother

\ifLuaTeX
\usepackage[bidi=basic]{babel}
\else
\usepackage[bidi=default]{babel}
\fi
\babelprovide[main,import]{vietnamese}
% get rid of language-specific shorthands (see #6817):
\let\LanguageShortHands\languageshorthands
\def\languageshorthands#1{}
\usepackage{bookmark}

\IfFileExists{xurl.sty}{\usepackage{xurl}}{} % add URL line breaks if available
\urlstyle{same} % disable monospaced font for URLs
\hypersetup{
  pdftitle={CÁC SỐ ĐẶC TRƯNG ĐO XU THẾ TRUNG TÂM CHO MẪU SỐ LIỆU GHÉP NHÓM},
  pdfauthor={ZO \textbar{} 2025-03-08},
  pdflang={vi},
  colorlinks=true,
  linkcolor={blue},
  filecolor={Maroon},
  citecolor={Blue},
  urlcolor={Blue},
  pdfcreator={LaTeX via pandoc}}


\title{CÁC SỐ ĐẶC TRƯNG ĐO XU THẾ TRUNG TÂM CHO MẪU SỐ LIỆU GHÉP NHÓM}
\usepackage{etoolbox}
\makeatletter
\providecommand{\subtitle}[1]{% add subtitle to \maketitle
  \apptocmd{\@title}{\par {\large #1 \par}}{}{}
}
\makeatother
\subtitle{BÀI TẬP TOÁN LỚP 11 \textbar{} SÁCH CÁNH DIỀU}
\author{ZO \textbar{} 2025-03-08}
\date{}

\begin{document}
\maketitle


Đây là các bài tập của Bài 1: \emph{Các số đặc trưng đo xu thế trung tâm
cho mẫu số liệu ghép nhóm}, Chương V: \emph{Một số yếu tố thống kê và
xác suất}, bộ \emph{Cánh Diều}, Toán 11 (trang 14, tập 2).

\section*{Bài tập 1}

Mẫu số liệu dưới đây ghi lại tốc độ của 40 ô-tô khi đi qua một trạm đo
tốc độ (đơn vị: km/h):

\begin{table}[!h]
\centering
\begin{tabular}{cccccccccc}
\toprule
48,5 & 43,0 & 50,0 & 55,0 & 45,0 & 60,0 & 53,0 & 55,5 & 44,0 & 65,0\\
51,0 & 62,5 & 41,0 & 44,5 & 57,0 & 57,0 & 68,0 & 49,0 & 46,5 & 53,5\\
61,0 & 49,5 & 54,0 & 62,0 & 59,0 & 56,0 & 47,0 & 50,0 & 60,0 & 61,0\\
49,5 & 52,5 & 57,0 & 47,0 & 60,0 & 55,0 & 45,0 & 47,5 & 48,0 & 61,5\\
\bottomrule
\end{tabular}
\end{table}

\begin{enumerate}
\def\labelenumi{\alph{enumi}.}
\tightlist
\item
  Lập bảng tần số ghép nhóm cho mẫu số liệu trên có sáu nhóm ứng với sáu
  nửa khoảng:
\end{enumerate}

\begin{center}
$[40; 45)$, $[45 ; 50)$, $[50; 55)$, $[55 ; 60)$, $[60; 65)$, $[65; 70)$.
\end{center}

\begin{enumerate}
\def\labelenumi{\alph{enumi}.}
\setcounter{enumi}{1}
\item
  Xác định trung bình cộng, trung vị, tứ phân vị của mẫu số liệu ghép
  nhóm trên.
\item
  Mốt của mẫu số liệu ghép nhóm trên là bao nhiêu?
\end{enumerate}

\begin{center}
\textbf{Lời giải}
\end{center}

\begin{enumerate}
\def\labelenumi{\alph{enumi}.}
\tightlist
\item
  Để đảm bảo phân nhóm chính xác, trước tiên mình sắp xếp số liệu theo
  thứ tự tăng dần.
\end{enumerate}

\begin{longtable*}{cccccccccc}
\toprule
\endfirsthead
\multicolumn{10}{@{}l}{\textit{(continued)}}\\
\toprule
\endhead

\endfoot
\bottomrule
\endlastfoot
41,0 & 43,0 & 44,0 & 44,5 & 45,0 & 45,0 & 46,5 & 47,0 & 47,0 & 47,5\\
48,0 & 48,5 & 49,0 & 49,5 & 49,5 & 50,0 & 50,0 & 51,0 & 52,5 & 53,0\\
53,5 & 54,0 & 55,0 & 55,0 & 55,5 & 56,0 & 57,0 & 57,0 & 57,0 & 59,0\\
60,0 & 60,0 & 60,0 & 61,0 & 61,0 & 61,5 & 62,0 & 62,5 & 65,0 & 68,0\\*
\end{longtable*}

Nhìn vào dãy có thứ tự tăng dần, bạn dễ dàng thấy rằng thuộc vào nhóm
\([40,45)\) có bốn số liệu bao gồm 41; 43; 44; 44,5. Hãy tiếp tục quan
sát để thu được bảng tần số ghép nhóm.

\begin{longtable*}{cc}
\toprule
Nhóm & Tần số\\
\midrule
\endfirsthead
\multicolumn{2}{@{}l}{\textit{(continued)}}\\
\toprule
Nhóm & Tần số\\
\midrule
\endhead

\endfoot
\bottomrule
\endlastfoot
\([40;45)\) & 4\\
\([45;50)\) & 11\\
\([50;55)\) & 7\\
\([55;60)\) & 8\\
\([60;65)\) & 8\\
\addlinespace
\([65;70)\) & 2\\
 & \(n=40\)\\*
\end{longtable*}

\begin{enumerate}
\def\labelenumi{\alph{enumi}.}
\setcounter{enumi}{1}
\tightlist
\item
  Mình thêm vào bảng tần số ghép nhóm ở trên cột số thứ tự để dễ quan
  sát, cột Giá trị đại diện để tính trung bình, và cột Tần số tích lũy
  để tính tứ phân vị.
\end{enumerate}

\begin{longtable*}{ccccc}
\toprule
  & Nhóm & Giá trị đại diện & Tần số & Tần số tích lũy\\
\midrule
\endfirsthead
\multicolumn{5}{@{}l}{\textit{(continued)}}\\
\toprule
  & Nhóm & Giá trị đại diện & Tần số & Tần số tích lũy\\
\midrule
\endhead

\endfoot
\bottomrule
\endlastfoot
1 & \([40;45)\) & 42,5 & 4 & 4\\
2 & \([45;50)\) & 47,5 & 11 & 15\\
3 & \([50;55)\) & 52,5 & 7 & 22\\
4 & \([55;60)\) & 57,5 & 8 & 30\\
5 & \([60;65)\) & 62,5 & 8 & 38\\
\addlinespace
6 & \([65;70)\) & 67,5 & 2 & 40\\
 &  &  & \(n=40\) & \\*
\end{longtable*}

\begin{itemize}
\tightlist
\item
  Mẫu có \(n=40\) số liệu. Nhóm một, \([40;45)\), có trung điểm
  \(x_1=42,5\) làm giá trị đại diện và có tần số \(n_1=4\). Ký hiệu
  tương tự cho các nhóm còn lại. Trung bình cộng là \begin{align*}
    \bar{x} 
        & = \frac{1}{n} (n_1\cdot x_1 + n_2\cdot x_2 + n_3\cdot x_3 + n_4\cdot x_4 + n_5\cdot x_5 + n_6\cdot x_6) \\
        & = \frac{1}{40} (4\cdot 42,5 + 11\cdot 47,5 + 7\cdot 52,5 + 8\cdot 57,5 + 8\cdot 62,5 + 2\cdot 67,5) \\
        & = \frac{431}{8} \\
        & = 53,875 \text{ (km/h)}.
  \end{align*}
\end{itemize}

\begin{itemize}
\tightlist
\item
  Nhóm ba, \([50;55)\), là nhóm đầu tiên có tần số tích lũy \(cf_3=22\)
  không nhỏ hơn \(\frac{n}{2}=20\). Nhóm này có đầu mút trái \(r=50\),
  độ dài \(d=5\) và tần số \(n_3=7\). Nhóm hai có tần số tích lũy
  \(cf_2=15\). Trung vị là \begin{align*}
        M_e
            & = r+\left(\frac{\frac{n}{2}-cf_2}{n_3}\right)\cdot d \\
            & = 50 + \left(\frac{20-15}{7}\right)\cdot 5 \\
            & = \frac{375}{7} \\
            & \approx 53,5714 \text{ (km/h).}
    \end{align*}
\end{itemize}

\begin{itemize}
\item
  Nhóm hai, \([45;50)\), là nhóm đầu tiên có tần số tích lũy \(cf_2=15\)
  không nhỏ hơn \(\frac{n}{4}=10\). Nhóm này có đầu mút trái \(s=45\),
  độ dài \(h=5\) và tần số \(n_2=11\). Nhóm một có tần số tích lũy là
  \(cf_1=4\). Tứ phân vị thứ nhất là \begin{align*}
        Q_1
            & = s + \left( \frac{\frac{n}{4}-cf_1}{n_2}\right)\cdot h \\
            & = 45 + \left(\frac{10-4}{11}\right)\cdot 5 \\
            & = \frac{525}{11} \\
            & \approx 47,7273 \text{ (km/h).}
    \end{align*}
\item
  Tứ phân vị thứ hai \(Q_2\) chính là trung vị \(M_e\), hay
  \(Q_2 = M_e = 53,5714\) (km/h).
\end{itemize}

\begin{itemize}
\tightlist
\item
  Nhóm bốn, \([55;60)\), là nhóm đầu tiên có tần số tích lũy \(cf_4=30\)
  bằng với \(\frac{3n}{4}=\frac{3\cdot 40}{4}=30\). Nhóm này có đầu mút
  trái \(t=55\), độ dài \(l=5\) và tần số \(n_4=8\). Nhóm thứ ba có tần
  số tích lũy là \(cf_3=22\). Tứ phân vị thứ ba là \begin{align*}
        Q_3
            & = t + \left(\frac{\frac{3n}{4}-cf_3}{n_4}\right)\cdot l \\
            & = 55 + \left(\frac{30-22}{8}\right)\cdot 5 \\
            & = 60 \text{ (km/h).}
    \end{align*}
\end{itemize}

\begin{enumerate}
\def\labelenumi{\alph{enumi}.}
\setcounter{enumi}{2}
\tightlist
\item
  Nhóm hai, \([45;50)\), là nhóm có tần số lớn nhất \(n_2=11\). Nó có
  đầu mút trái \(u=45\) và độ dài \(g=5\). Nhóm một có tần số \(n_1=4\)
  và nhóm ba có tần số \(n_3=7\). Mốt là \begin{align*}
       M_o
           & = u + \left(\frac{n_2-n_1}{2n_2-n_1-n_3}\right) \cdot g \\
           & = 45 + \left(\frac{11-4}{2\cdot 11-4-7}\right)\cdot 5 \\
           & = \frac{530}{11} \\
           & \approx 48,1818 \text{ (km/h)}.
   \end{align*}
\end{enumerate}

Tóm lại, bảng tần số ghép nhóm là

\begin{center}
\centering
\begin{tabular}{|c|c|c|c|c|c|}
\hline 
$[40; 45)$ & $[45 ; 50)$ & $[50; 55)$ & $[55 ; 60)$ & $[60; 65)$ & $[65; 70)$ \\
\hline 
4 & 11 & 7 & 8 & 8 & 2. \\
\hline
\end{tabular}
\end{center}

Các số đặc trưng đo xu thế trung tâm là

\begin{center}
\begin{tabular}{|c|c|c|c|c|}
\hline
$\bar{x}$ & $Q_1$ & $Q_2$ ($M_e$) & $Q_3$ & $M_o$ \\
\hline
53,875 & 47,7273 & 53,5714 & 60 & 48,1818. \\
\hline 
\end{tabular}
\end{center}

\section*{Bài tập 2}

Mẫu số liệu sau ghi lại cân nặng của 30 bạn học sinh (đơn vị: kg):

\begin{longtable*}{cccccccccc}
\toprule
\endfirsthead
\multicolumn{10}{@{}l}{\textit{(continued)}}\\
\toprule
\endhead

\endfoot
\bottomrule
\endlastfoot
17,0 & 40,0 & 39,0 & 40,5 & 42,0 & 51,0 & 41,5 & 39,0 & 41,0 & 30,0\\
40,0 & 42,0 & 40,5 & 39,5 & 41,0 & 40,5 & 37,0 & 39,5 & 40,0 & 41,0\\
38,5 & 39,5 & 40,0 & 41,0 & 39,0 & 40,5 & 40,0 & 38,5 & 39,5 & 41,5\\*
\end{longtable*}

\begin{enumerate}
\def\labelenumi{\alph{enumi}.}
\tightlist
\item
  Lập bảng tần số ghép nhóm cho mẫu số liệu trên có tám nhóm ứng với tám
  nửa khoảng:
\end{enumerate}

\begin{center}
$[15; 20)$, $[20; 25)$, $[25; 30)$, $[30; 35)$, $[35; 40)$, $[40 ; 45)$, $[45; 50)$, $[50; 55)$.
\end{center}

\begin{enumerate}
\def\labelenumi{\alph{enumi}.}
\setcounter{enumi}{1}
\item
  Xác định trung bình cộng, trung vị, tứ phân vị của mẫu số liệu ghép
  nhóm trên.
\item
  Mốt của mẫu số liệu ghép nhóm trên là bao nhiêu?
\end{enumerate}

\begin{center}
\textbf{Lời giải}
\end{center}

\begin{enumerate}
\def\labelenumi{\alph{enumi}.}
\tightlist
\item
  Để đảm bảo phân nhóm chính xác, trước tiên mình sắp xếp số liệu theo
  thứ tự tăng dần.
\end{enumerate}

\begin{longtable*}{cccccccccc}
\toprule
\endfirsthead
\multicolumn{10}{@{}l}{\textit{(continued)}}\\
\toprule
\endhead

\endfoot
\bottomrule
\endlastfoot
17,0 & 30,0 & 37,0 & 38,5 & 38,5 & 39,0 & 39,0 & 39,0 & 39,5 & 39,5\\
39,5 & 39,5 & 40,0 & 40,0 & 40,0 & 40,0 & 40,0 & 40,5 & 40,5 & 40,5\\
40,5 & 41,0 & 41,0 & 41,0 & 41,0 & 41,5 & 41,5 & 42,0 & 42,0 & 51,0\\*
\end{longtable*}

Nhìn vào dãy có thứ tự tăng dần, bạn dễ dàng thấy rằng thuộc vào nhóm
\([15,20)\) chỉ có một số liệu là 17. Hãy tiếp tục quan sát để thu được
bảng tần số ghép nhóm.

\begin{longtable*}{cc}
\toprule
Nhóm & Tần số\\
\midrule
\endfirsthead
\multicolumn{2}{@{}l}{\textit{(continued)}}\\
\toprule
Nhóm & Tần số\\
\midrule
\endhead

\endfoot
\bottomrule
\endlastfoot
\([15;20)\) & 1\\
\([20;25)\) & 0\\
\([25;30)\) & 0\\
\([30;35)\) & 1\\
\([35;40)\) & 10\\
\addlinespace
\([40;45)\) & 17\\
\([45;50)\) & 0\\
\([50;55)\) & 1\\
 & \(n=40\)\\*
\end{longtable*}

\begin{enumerate}
\def\labelenumi{\alph{enumi}.}
\setcounter{enumi}{1}
\tightlist
\item
  Mình thêm vào bảng tần số ghép nhóm ở trên cột số thứ tự để dễ quan
  sát, cột Giá trị đại diện để tính trung bình, và cột Tần số tích lũy
  để tính tứ phân vị.
\end{enumerate}

\begin{longtable*}{ccccc}
\toprule
  & Nhóm & Giá trị đại diện & Tần số & Tần số tích lũy\\
\midrule
\endfirsthead
\multicolumn{5}{@{}l}{\textit{(continued)}}\\
\toprule
  & Nhóm & Giá trị đại diện & Tần số & Tần số tích lũy\\
\midrule
\endhead

\endfoot
\bottomrule
\endlastfoot
1 & \([15;20)\) & 17,5 & 1 & 1\\
2 & \([20;25)\) & 22,5 & 0 & 1\\
3 & \([25;30)\) & 27,5 & 0 & 1\\
4 & \([30;35)\) & 32,5 & 1 & 2\\
5 & \([35;40)\) & 37,5 & 10 & 12\\
\addlinespace
6 & \([40;45)\) & 42,5 & 17 & 29\\
7 & \([45;50)\) & 47,5 & 0 & 29\\
8 & \([50;55)\) & 52,5 & 1 & 30\\
 &  &  & \(n=40\) & \\*
\end{longtable*}

\begin{itemize}
\tightlist
\item
  Mẫu có \(n=30\) số liệu. Nhóm một, \([15;20)\), có trung điểm
  \(x_1=17,5\) làm giá trị đại diện và có tần số \(n_1=1\). Ký hiệu
  tương tự với các nhóm còn lại. Trung bình là \begin{align*}
  \bar{x}
    & = \frac{1}{n}(n_1\cdot x_1 + n_2\cdot x_2 + n_3\cdot x_3 + n_4\cdot x_4 + n_5\cdot x_5 + n_6\cdot x_6 + n_7\cdot x_7 + n_8\cdot x_8) \\
    & = \frac{1}{30} (1\cdot 17,5 + 0\cdot 22,5 + 0\cdot 27,5+ 1\cdot 32,5 + 10\cdot 37,5 + 17\cdot 42,5 + 0\cdot 47,5 + 52,5 \cdot 1) \\
    & = 40 \text{ (kg).}
  \end{align*}
\end{itemize}

\begin{itemize}
\tightlist
\item
  Nhóm sáu, \([40;45)\), là nhóm đầu tiên có tần số tích lũy \(cf_6=29\)
  không nhỏ hơn \(\frac{n}{2}=15\). Nhóm này có đầu mút trái \(r=40\),
  độ dài \(d=5\) và tần số \(n_6=17\). Nhóm năm liền trước nó có tần số
  tích lũy \(cf_5=12\). Trung vị là \begin{align*}
  M_e
    & = r + \left(\frac{\frac{n}{2}-cf_5}{n_6}\right)\cdot d \\
    & = 40 + \left(\frac{15-12}{17}\right)\cdot 5 \\
    & = \frac{695}{17} \\
    & \approx 40,8824 \text{ (kg).}
  \end{align*}
\end{itemize}

\begin{itemize}
\item
  Nhóm năm \([35;40)\) là nhóm đầu tiên có tần số tích lũy \(cf_5=12\)
  không nhỏ hơn \(\frac{n}{4}=7,5\). Nhóm này có đầ mút trái \(s=35\),
  độ dài \(h=5\) và tần số \(n_5=10\). Nhóm bốn liền trước nó có tần số
  tích lũy \(cf_4=2\). Tứ phân vị thứ nhất là \begin{align*}
  Q_1
    & = s + \left( \frac{\frac{n}{4}-cf_4}{n_5}\right)\cdot h \\
    & = 35 + \left(\frac{7,5-2}{10}\right)\cdot 5 \\
    & = \frac{151}{4} \\
    & = 37,75 \text{ (kg).}
  \end{align*}
\item
  Tứ phân vị thứ hai \(Q_2\) chính là trung vị \(M_e\), hay
  \(Q_2=M_e \approx 40,8824\) (kg).
\end{itemize}

\begin{itemize}
\tightlist
\item
  Nhóm sáu \([40;45)\) là nhòm đầu tiên có tần số tích lũy \(cf_6=29\)
  không nhỏ hơn \(\frac{3n}{4}=22,5\). Nhóm này có đầu mút trái
  \(t=40\), độ dài \(l=5\) và tần số \(n_6=17\). Nhóm năm liền trước nó
  có tần số tích lũy là \(cf_5=12\). Tứ phân vị thứ ba là \begin{align*}
  Q_3
    & = t + \left(\frac{\frac{3n}{4}-cf_5}{n_6}\right)\cdot l \\
    & = 40 +\left(\frac{22,5-12}{17}\right) \cdot 5 \\
    & = \frac{1465}{34} \\
    & \approx 43,0882 \text{ (kg).} 
  \end{align*}
\end{itemize}

\begin{enumerate}
\def\labelenumi{\alph{enumi}.}
\setcounter{enumi}{2}
\tightlist
\item
  Nhóm sáu \([40,45)\) là nhóm có tần số lớn nhất \(n_6=17\). Nhóm này
  có đầu mút trái \(u=40\) và độ dài \(g=5\). Nhóm năm liền trước nó có
  tần số \(n_5=10\) và nhóm bảy liền sau nó có tần số \(n_7=0\). Mốt là
  \begin{align*}
  M_o 
   & = u + \left(\frac{n_6-n_5}{2n_6 - n_5 - n_7}\right)\cdot g \\
   & = 40 + \left(\frac{17-10}{2\cdot 17 - 10 - 0}\right)\cdot 5 \\
   & = \frac{995}{24} \\
   & \approx 41,5 \text{ (kg).}
  \end{align*}
\end{enumerate}

Tóm lại, bảng tần số ghép nhóm là

\begin{center}
\begin{tabular}{|c|c|c|c|c|c|c|c|}
\hline 
$[15;20)$ & $[20; 25)$ & $[25; 30)$ & $[30; 35)$ & $[35;40)$ & $[40; 45)$ & $[45;50)$ & $[50;55)$ \\
\hline 
1 & 0 & 0 & 1 & 10 & 17 & 0 & 1. \\
\hline
\end{tabular}
\end{center}

Các số đặc trưng đo xu thế trung tâm là

\begin{center}
\begin{tabular}{|c|c|c|c|c|}
\hline
$\bar{x}$ & $Q_1$ & $Q_2$ ($M_e$) & $Q_3$ & $M_o$ \\
\hline
40 & 37,75 & 40,8824 & 43,0882 & 41,4583. \\
\hline 
\end{tabular}
\end{center}

\section*{Bài tập 3}

Bảng 15 cho ta bảng tần số ghép nhóm số liệu thống kê chiều cao của 40
mẫu cây ở một vườn thực vật (đơn vị: cm).

\begin{enumerate}
\def\labelenumi{\alph{enumi}.}
\item
  Xác định trung bình cộng, trung vị, tứ phân vị của mẫu số liệu ghép
  nhóm trên.
\item
  Mốt của mẫu số liệu ghép nhóm trên là bao nhiêu?
\end{enumerate}

\begin{longtable*}{ccc}
\toprule
Nhóm & Tần số & Tần số tích lũy\\
\midrule
\endfirsthead
\multicolumn{3}{@{}l}{\textit{(continued)}}\\
\toprule
Nhóm & Tần số & Tần số tích lũy\\
\midrule
\endhead

\endfoot
\bottomrule
\endlastfoot
\([30;40)\) & 4 & 4\\
\([40;50)\) & 10 & 14\\
\([50;60)\) & 14 & 28\\
\([60;70)\) & 6 & 34\\
\([70;80)\) & 4 & 38\\
\addlinespace
\([80;90)\) & 2 & 40\\
 & \(n=40\) & \\*
\end{longtable*}

\begin{center}
\textbf{Lời giải}
\end{center}

\begin{enumerate}
\def\labelenumi{\alph{enumi}.}
\tightlist
\item
  Mình sẽ thêm vào bảng đã cho cột Giá trị đại diện để tính trung bình,
  và cột số thứ tự để dễ quan sát.
\end{enumerate}

\begin{longtable*}{ccccc}
\toprule
  & Nhóm & Giá trị đại diện & Tần số & Tần số tích lũy\\
\midrule
\endfirsthead
\multicolumn{5}{@{}l}{\textit{(continued)}}\\
\toprule
  & Nhóm & Giá trị đại diện & Tần số & Tần số tích lũy\\
\midrule
\endhead

\endfoot
\bottomrule
\endlastfoot
1 & \([30;40)\) & 35 & 4 & 4\\
2 & \([40;50)\) & 45 & 10 & 14\\
3 & \([50;60)\) & 55 & 14 & 28\\
4 & \([60;70)\) & 65 & 6 & 34\\
5 & \([70;80)\) & 75 & 4 & 38\\
\addlinespace
6 & \([80;90)\) & 85 & 2 & 40\\
 &  &  & \(n=40\) & \\*
\end{longtable*}

\begin{itemize}
\tightlist
\item
  Mẫu có \(n=40\) số liệu. Nhóm một có giá trị đại diện \(x_1=35\) và
  tần số \(n_1=4\), vân vân cho đến nhóm sáu có giá trị đại diện
  \(x_6=85\) và tần số \(n_6=2\). Trung bình là
\end{itemize}

\begin{align*}
    \bar{x}
        & = \frac{1}{n}(n_1\cdot x_1 + n_2\cdot x_2 + n_3\cdot x_3 + n_4\cdot x_4 + n_5\cdot x_5 + n_6\cdot x_6) \\
        & = \frac{1}{40} (4\cdot 35 + 10\cdot 45 + 14\cdot 55 + 6\cdot 65 + 4\cdot 75 + 2\cdot 85) \\
        & = \frac{111}{2} \\
        & = 55,5 \text{ (cm).}
\end{align*}

\begin{itemize}
\tightlist
\item
  Nhóm ba, \([50;55)\), là nhóm đầu tiên có tần số tích lũy \(cf_3=28\)
  không nhỏ hơn \(\frac{n}{2}=20\). Nhóm này có đầu mút trái \(r=50\),
  độ dài \(d=10\) và tần số \(n_3=14\). Nhóm hai liền trước nó có tần số
  tích lũy \(cf_2=14\). Trung vị là
\end{itemize}

\begin{align*}
M_e
    & = r + \left(\frac{\frac{n}{2}-cf_2}{n_3}\right)\cdot d \\
    & = 50 + \left(\frac{20-14}{14}\right)\cdot 10 \\
    & = \frac{380}{7} \\
    & \approx 54,2857 \text{ (cm).}
\end{align*}

\begin{itemize}
\item
  Nhóm hai, \([40;50)\), là nhóm đầu tiên có tần số tích lũy \(cf_2=14\)
  không nhỏ hơn \(\frac{n}{4}=10\). Nhóm này có đầu mút trái \(s=40\),
  độ dài \(h=10\) và tần số \(n_2=10\). Nhóm một có tần số tích lũy là
  \(cf_1=4\). Tứ phân vị thứ nhất là

  \begin{align*}
        Q_1
            & = s + \left( \frac{\frac{n}{4}-cf_1}{n_2}\right)\cdot h \\
            & = 40 + \left(\frac{10-4}{10}\right)\cdot 10 \\
            & = 46 \text{ (cm).}
    \end{align*}
\item
  Tứ phân vị thứ hai \(Q_2\) chính là trung vị \(M_e\), hay
  \(Q_2 = M_e = 54,2857\) (cm).
\end{itemize}

\begin{itemize}
\item
  Nhóm bốn, \([60;70)\), là nhóm đầu tiên có tần số tích lũy \(cf_4=34\)
  không nhỏ hơn \(\frac{3n}{4}=30\). Nhóm này có đầu mút trái \(t=60\),
  độ dài \(l=10\) và tần số \(n_4=6\). Nhóm thứ ba có tần số tích lũy là
  \(cf_3=28\). Tứ phân vị thứ ba là

  \begin{align*}
        Q_3
            & = t + \left(\frac{\frac{3n}{4}-cf_3}{n_4}\right)\cdot l \\
            & = 60 + \left(\frac{30-28}{6}\right)\cdot 10 \\
            & = \frac{190}{3} \\
            & = 63,3333 \text{ (cm).}
    \end{align*}
\end{itemize}

\begin{enumerate}
\def\labelenumi{\alph{enumi}.}
\setcounter{enumi}{1}
\tightlist
\item
  Nhóm ba, \([50;60)\), là nhóm có tần số lớn nhất \(n_3=14\). Nó có đầu
  mút trái \(u=50\) và độ dài \(g=10\). Nhóm hai có tần số \(n_2=10\) và
  nhóm bốn có tần số \(n_4=6\). Mốt là \begin{align*}
       M_o
           & = u + \left(\frac{n_3-n_2}{2n_3-n_2-n_4}\right) \cdot g \\
           & = 50 + \left(\frac{14-10}{2\cdot 14-10-6}\right)\cdot 10 \\
           & = \frac{160}{3} \\
           & \approx 53,3333 \text{ (cm)}.
   \end{align*}
\end{enumerate}

Tóm lại, bảng tần số ghép nhóm là

\begin{center}
\begin{tabular}{|c|c|c|c|c|c|}
\hline 
$[30;40)$ & $[40;50)$ & $[50;60)$ & $[60;70)$ & $[70;80)$ & $[80;90)$  \\
\hline 
1 & 0 & 0 & 1 & 10 & 17. \\
\hline
\end{tabular}
\end{center}

Các số đặc trưng đo xu thế trung tâm là

\begin{center}
\begin{tabular}{|c|c|c|c|c|}
\hline
$\bar{x}$ & $Q_1$ & $Q_2$ ($M_e$) & $Q_3$ & $M_o$ \\
\hline
55,5 & 46 & 54,2857 & 63,3333 & 53,3333. \\
\hline 
\end{tabular}
\end{center}




\end{document}
