\documentclass[border=10pt]{standalone}
\usepackage{tikz}
\usetikzlibrary{arrows.meta}
\begin{document}
\begin{tikzpicture}[
  x=1.5cm, y=1cm, % Kéo ngang, nén dọc
% >=Stealth, thick,
  every node/.style={font=\scriptsize, inner sep=1pt}
]
  % Trục x
  \draw[<->] (-2.5,0) -- (2.5,0) node[above right] {$x$};
  \draw[<->] (0,-1) -- (0,5) node[above left] {$y$};
  \filldraw (0,0) circle (1pt) node[below left, yshift=-2pt] {$O$};

  % Các đoạn thẳng
  \foreach \x in {-2, -1.5, -1, -.5, 0, .5, 1, 1.5, 2}{
    \pgfmathsetmacro{\y}{\x*\x}
    \draw[dashed] 
      (\x,0)
        % Nếu x khác 0 thì mới ghi nhãn
        node[below, yshift=-2pt] {\ifdim\x pt=0pt\relax\else\pgfmathprintnumber[fixed, precision=1, use comma]{\x}\fi}
      --
      (\x,\y) 
      -- (0, \y) node[above left] {\ifdim\y pt=0pt\relax\else\pgfmathprintnumber[fixed, precision=2, use comma]{\y}\fi};
    \filldraw (\x,\y) circle (1pt);
  }

  % Vẽ các đoạn thẳng, điểm, nhãn
\draw[smooth]
  plot coordinates {
    (-2,4) (-1.5,2.25) (-1,1) (-0.5,0.25)
    (0,0) (0.5,0.25) (1,1) (1.5,2.25) (2,4)
  };
\end{tikzpicture}
\end{document}
