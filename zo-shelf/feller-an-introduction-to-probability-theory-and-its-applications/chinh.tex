\documentclass[12pt]{book} % Định dạng tài liệu, kích thước chữ 12pt

\usepackage{polyglossia} % Quản lí ngôn ngữ
\setdefaultlanguage{vietnamese}
\setotherlanguages{english}
\usepackage{fontspec} % Cung cấp khả năng sử dụng phông chữ OpenType và TrueType
\usepackage{
    amsmath, % Các lệnh toán học
    amsfonts, % Các kí hiệu toán học
    amssymb % Các kí hiệu toán học
}
\usepackage{unicode-math} % Cung cấp hỗ trợ cho các phông chữ toán học Unicode
\setmainfont{STIX Two Text} % Thiết lập phông chữ chính là STIX Two Text
\setmathfont{STIX Two Math} % Thiết lập phông chữ toán học là STIX Two Math

\usepackage[a4paper, left=2cm, right=2cm, top=2cm, bottom=2cm]{geometry} % Định dạng kích thước và lề trang

\usepackage{fancyhdr} % Hiệu chỉnh header và footer
\pagestyle{fancy}
\fancyhf{}
%\fancyhead[LE]{\nouppercase{\rightmark\hfill\leftmark}}
%\fancyhead[RO]{\nouppercase{\leftmark\hfill\rightmark}}
%\fancyhead[CO]{\footnotesize\nouppercase\rightmark} % O là lẻ
%\fancyhead[CE]{\footnotesize\nouppercase\leftmark}
%\fancyfoot[C]{\thepage}
\fancyhf[HCO]{\footnotesize\nouppercase\rightmark} % O là lẻ
\fancyhf[HCE]{\footnotesize\nouppercase\leftmark}
\fancyhf[FC]{\thepage}
\renewcommand{\headrulewidth}{0pt}

\usepackage{graphicx} % Hỗ trợ chèn hình ảnh vào tài liệu

\usepackage{xcolor} % Gói màu sắc để tùy chỉnh màu sắc

\usepackage[vietnamese]{hyperref} % Tạo liên kết và tham chiếu trong tài liệu
\hypersetup{
    colorlinks=true, % Kích hoạt màu sắc cho các liên kết
    linkcolor=darkgray, % Màu của liên kết nội bộ
    citecolor=blue, % Màu của liên kết tham chiếu
    filecolor=blue, % Màu của liên kết tập tin
    urlcolor=blue % Màu của liên kết URL
}
\renewcommand{\sectionautorefname}{Mục} % Đổi tên tự động của các liên kết phần mục từ "Section" thành "Mục"

\usepackage{bookmark} % Tạo mục lục nhanh và chính xác hơn

\usepackage[backend=biber,style=authoryear,sorting=none]{biblatex} % Quản lí tài liệu tham khảo với biblatex và biber
\addbibresource{references.bib} % Thêm tài liệu tham khảo từ tệp references.bib
\DeclareLanguageMapping{vietnamese}{vietnamese-english} % Định nghĩa ánh xạ ngôn ngữ cho tiếng Việt
\DeclareLanguageMapping{english}{english} % Định nghĩa ánh xạ ngôn ngữ cho tiếng Anh

\newcounter{myproblem} % Tạo một bộ đếm mới cho môi trường myproblem
\newenvironment{myproblem}[1][]{%
    \vspace{10pt} % Khảng cách từ trên xuống
    \noindent\textsc{Bài tập #1} % Định dạng tiêu đề môi trường myproblem
    \noindent
}{%
    \par
    \vspace{10pt} % Khoảng cách từ dưới lên
}
\newenvironment{mydotproblem}[1][]{%
    \vspace{10pt}
    \noindent\textsc{Bài tập. #1}
    \noindent
}{%
    \par
    \vspace{10pt} 
}
\newenvironment{mynumproblem}[1][]{%
    \refstepcounter{myproblem}
    \vspace{10pt}
    \noindent\textsc{Bài tập \theproblem. #1}
    \noindent
}{%
    \par
    \vspace{10pt}
}

\usepackage{ntheorem} % Thiet lap cho moi truong tua dinh nghia
\theoremstyle{nonumberplain}
\theorembodyfont{\upshape}
\newtheorem{vidu}{Ví dụ}
%\theoremstyle{nonumberplain}
%\theoremheaderfont{\scshape}
%\newtheorem{survey}{Khảo sát}
%\newtheorem{guide}{Hướng dẫn}
%\newtheorem{essay}{Luận}
%\newtheorem{remark}{Nhận xét}
%\newtheorem{gloss}{Chú thích}
%\newtheorem{hint}{Gợi ý}

\title{Nhập môn Lí thuyết xác suất và Ứng dụng của nó \\ An Introduction to Probability Theory and Its Applications (Third Edition)}
\author{William Feller}
\date{\today}

\allowdisplaybreaks % Cho phép ngắt dòng trong các công thức toán học dài

\begin{document}

\maketitle
% \tableofcontents
\chapter{Dẫn nhập: Bản chất của lí thuyết xác suất}

\section{Nền tảng}
{\bf Xác suất là một lĩnh vực toán học có mục tiêu tương tự như, ví dụ, hình học hoặc cơ học giải tích. Trong mỗi lĩnh vực, chúng ta cần phân biệt rõ ba khía cạnh của lí thuyết: (a) nội dung lí luận hình thức, (b) nền tảng trực giác, (c) các ứng dụng. Bản chất và sức hấp dẫn của toàn bộ cấu trúc không thể được đánh giá đúng nếu không xem xét cả ba khía cạnh này trong mối quan hệ thích hợp với chúng.}

\noindent\rule{\textwidth}{.4pt} 

Đoạn văn này so sánh xác suất với các lĩnh vực toán học khác như hình học và cơ học giải tích. Điểm chung giữa các lĩnh vực này là mỗi lĩnh vực đều cần được hiểu từ ba khía cạnh khác nhau:
\begin{enumerate}
    \item Nội dung Logic hình thức (formal logical content): Đây là phần logic và lý thuyết thuần túy của xác suất, bao gồm các định lý, quy tắc, và chứng minh. Đây là những yếu tố chính xác, có cấu trúc chặt chẽ và có thể được kiểm chứng qua lý luận toán học. 
    
    Một ví dụ cụ thể là định lí Bayes:
    \begin{equation*}
        P(A\vert B)=\frac{P(B\vert A)P(A)}{P(B)}.
    \end{equation*}
    Đây là công thức chính xác để tính xác suất có điều kiện, cho phép chúng ta xác định xác suất sự kiện $A$ xảy ra khi đã biết sự kiện $B$ xảy ra.
    \item Nền tảng trực giác (intuitive background): Phần này liên quan đến cách chúng ta hiểu và hình dung các khái niệm toán học. Trong xác suất, điều này bao gồm trực giác về cách các sự kiện ngẫu nhiên xảy ra, hoặc ý niệm về khả năng xảy ra của một sự kiện mà không cần đến tính toán cụ thể. 
    
    Ví dụ, tình huống có hai túi bi. Một túi chứa 3 bi đỏ và 7 bi xanh, túi kia chứa 4 bi đỏ và 6 bi xanh. Định lí Bayes sẽ giúp chúng ta tính toán xác suất để chọn được một bi từ túi nào dựa trên màu sắc của bi đã chọn. Trực giác sẽ cho ta một cảm giác rằng túi có nhiều bi xanh hơn thì khả năng chọn được bi xanh sẽ cao hơn. Chúng ta không cần phải biết chính xác công thức, nhưng trực giác này đúng về mặt lí thuyết.

    \item Các ứng dụng (applications): Đây là phần thực tế, nơi các lý thuyết toán học được áp dụng để giải quyết các vấn đề trong thực tế. Trong xác suất, điều này có thể bao gồm các ứng dụng trong thống kê, tài chính, vật lý, hoặc các lĩnh vực khác.
\end{enumerate}
Điểm quan trọng của đoạn văn là để hiểu đầy đủ một lý thuyết toán học (như xác suất), chúng ta không thể chỉ tập trung vào một khía cạnh duy nhất, mà phải xem xét cả ba khía cạnh này và cách chúng liên hệ với nhau. Sự kết hợp giữa lý thuyết, trực giác và ứng dụng là yếu tố tạo nên sức hấp dẫn và ý nghĩa của toàn bộ cấu trúc lý thuyết.

\begin{vidu}
    Một ví dụ thực tế đã diễn ra và sử dụng Định lý Bayes là trong việc chẩn đoán bệnh ung thư vú dựa trên kết quả xét nghiệm sàng lọc bằng chụp nhũ ảnh (mammogram).
\paragraph{Bối cảnh}
\begin{itemize}
    \item Mammogram là một xét nghiệm phổ biến để phát hiện sớm ung thư vú.
    \item Tuy nhiên, kết quả dương tính (test cho thấy có dấu hiệu của ung thư) không phải lúc nào cũng chính xác, vì có thể xảy ra dương tính giả (false positive) - khi kết quả dương tính nhưng người đó không thực sự mắc bệnh.
\end{itemize}
\paragraph{Dữ liệu}
\begin{enumerate}
    \item Tỷ lệ phụ nữ bị ung thư vú trong dân số ($P(C)$): Giả sử tỷ lệ này là $1\%(0.01)$.
    \item Xác suất xét nghiệm dương tính khi phụ nữ mắc bệnh ($P(T\vert C)$): Giả sử mammogram cho kết quả dương tính với $80\%$ độ chính xác ($0.8$).
    \item Xác suất xét nghiệm dương tính khi phụ nữ không bị ung thư ($P(T\vert \lnot C)$): Độ chính xác không hoàn hảo của mammogram tạo ra tỷ lệ dương tính giả khoảng $10\% (0.1)$.
\end{enumerate}
\paragraph{Câu hỏi:} Giả sử một phụ nữ nhận được kết quả dương tính, xác suất để cô ấy thực sự bị ung thư vú là bao nhiêu?
\paragraph{Sử dụng Định lý Bayes}
Chúng ta cần tính $P(C\vert T)$ - xác suất người đó mắc ung thư vú khi xét nghiệm dương tính. Định lý Bayes cho chúng ta công thức:
\begin{equation*}
    P(C\vert T)=\frac{P(T\vert C)P(C)}{P(T)}
\end{equation*}
trong đó:
\begin{itemize}
    \item $P(T\vert C)=0.8$ là xác suất xét nghiệm dương tính khi người đó có bệnh.
    \item $P(C)=0.01$ là xác suất mắc bệnh (trong toàn dân số).
    \item P(T) là xác suất tổng thể của kết quả xét nghiệm dương tính, và có thể được tính bằng công thức:
    \begin{equation*}
        P(T)=P(T\vert C)P(C)+P(T\vert \lnot C)P(\lnot C).
    \end{equation*}
    \item $P(T\vert \lnot C)=0.1$ là xác suất dương tính giả.
    \item $P(\lnot C)=1-P(C)=0.99$ là xác suất không mắc bệnh.
\end{itemize}
\paragraph{Tính toán}
\begin{enumerate}
    \item Tính $P(T)$:
    \item \begin{equation*}
        P(T)=(0.8\cdot 0.01)+(0.1\cdot 0.99)=0.107
    \end{equation*}
    \item Tính $P(C\vert T)$:
    \begin{equation*}
    P(C\vert T)=\frac{0.8\cdot 0.01}{0.107}\approx 0,0748.
    \end{equation*}
\end{enumerate}
\paragraph{Kết quả:} Xác suất người phụ nữ thực sự mắc ung thư vú sau khi có kết quả xét nghiệm dương tính là khoảng $7.5\%$. Điều này nghĩa là mặc dù xét nghiệm dương tính, khả năng thực sự mắc bệnh của người đó vẫn rất thấp, vì tỷ lệ ung thư vú trong dân số là rất nhỏ và có khả năng dương tính giả.
\paragraph{Ý nghĩa của Định lý Bayes trong thực tế:} Kết quả này minh họa rõ ràng việc sử dụng Định lý Bayes để kết hợp thông tin từ xét nghiệm với tỷ lệ bệnh trong dân số, giúp các bác sĩ và bệnh nhân hiểu rõ hơn về kết quả xét nghiệm và tránh hiểu nhầm rằng một kết quả dương tính đồng nghĩa với khả năng mắc bệnh rất cao.

Trong thực tế, định lý này đã được áp dụng rộng rãi trong nhiều lĩnh vực y tế, giúp các chuyên gia y tế đưa ra các quyết định chính xác hơn dựa trên xác suất thực tế, chứ không chỉ dựa vào kết quả xét nghiệm ban đầu.
\end{vidu}
    
\subsection*{Minh họa ba khía cạnh}
Để minh họa trực quan cho ba khía cạnh của lý thuyết xác suất (hoặc bất kỳ lý thuyết toán học nào), bạn có thể sử dụng một biểu đồ hình tam giác hoặc sơ đồ Venn, giúp thể hiện mối quan hệ giữa các khía cạnh ``nội dung logic'', ``trực giác'' và ``ứng dụng''. Dưới đây là hai gợi ý cho hình minh họa:
\begin{enumerate}
    \item Biểu đồ hình tam giác:  Trong biểu đồ này, mỗi đỉnh của tam giác sẽ đại diện cho một khía cạnh của lý thuyết. Mối quan hệ giữa các khía cạnh sẽ được thể hiện qua ba cạnh nối các đỉnh lại với nhau.
    \begin{itemize}
        \item Đỉnh trên cùng A: Nội dung logic hình thức
        \item Đỉnh trái dưới B: Nền tảng trực giác
        \item Đỉnh phải dưới C: Các ứng dụng
    \end{itemize}
    Trên mỗi cạnh của tam giác, bạn có thể chú thích về mối liên hệ giữa hai khía cạnh. Ví dụ:
    \begin{itemize}
        \item Cạnh AB: Mối quan hệ giữa lý thuyết logic và trực giác (làm thế nào để trực giác dẫn đến các định lý chính xác).
        \item Cạnh BC: Cách trực giác dẫn đến các ứng dụng thực tế.
        \item Cạnh AC: Cách lý thuyết logic dẫn đến các ứng dụng mà không cần quá nhiều trực giác.
    \end{itemize}

    \item Sơ đồ Venn: Bạn có thể sử dụng ba vòng tròn giao nhau, mỗi vòng đại diện cho một khía cạnh. Phần giao giữa ba vòng tròn thể hiện sự kết hợp của cả ba khía cạnh:
    \begin{itemize}
        \item Vòng tròn 1: Nội dung logic hình thức
        \item Vòng tròn 2: Nền tảng trực giác
        \item Vòng tròn 3: Các ứng dụng
    \end{itemize}
    Phần giao nhau của các vòng thể hiện những điểm mà cả ba khía cạnh cùng tồn tại, nhấn mạnh rằng để hiểu đầy đủ lý thuyết, ta cần sự hòa quyện giữa cả ba khía cạnh. Gợi ý chú thích cho các phần giao nhau:
    \begin{itemize}
        \item Logic và Trực giác: Những lý thuyết có thể được chứng minh nhưng cũng dễ hiểu thông qua trực giác.
        \item Trực giác và Ứng dụng: Những ứng dụng thực tế có thể được hiểu một cách tự nhiên mà không cần đến các lý thuyết quá phức tạp.
        \item Logic và Ứng dụng: Các ứng dụng có thể được xây dựng từ các định lý toán học mà không cần nhiều đến trực giác.
    \end{itemize}
\end{enumerate}
Cả hai cách vẽ đều giúp thể hiện rõ ràng ba khía cạnh và mối liên hệ của chúng trong việc hình thành một lý thuyết hoàn chỉnh.

\begin{figure}[htbp]
    \centering
    \includegraphics*[scale=1]{ba_khia_canh_li_thuyet_xac_suat.pdf}
    \caption{Ba khía cạnh của lí thuyết xác suất.}
    \label{fig:ba_khia_canh_li_thuyet_xac_suat}
\end{figure}

% \begin{figure}[htbp]
%     \centering
%     \includegraphics*[scale=1]{images/ba-khia-canh-li-thuyet-xac-suat/ba_khia_canh_li_thuyet_xac_suat.pdf}
%     \caption{Ba khía cạnh của lí thuyết xác suất.}
%     \label{fig:ba_khia_canh_li_thuyet_xac_suat}
% \end{figure}

\noindent\rule{\textwidth}{.4pt}

\noindent\rule{\textwidth}{.4pt}

\section*{(a) Nội dung lí luận hình thức}
{\bf Về mặt tiên đề, toán học chỉ quan tâm đến các mối quan hệ giữa những điều không được định nghĩa. Khía cạnh này được minh họa rõ ràng qua trò chơi cờ vua. Không thể ``dinh5 nghĩa'' cờ vua theo cách nào khác ngoài việc mêu ra một bộ qui tắc. Hình dạng truyền thống của các quân cờ có thể được mô tả phần nào, nhưng không phải lúc nào cũng rõ ràng quân nào là ``vua''. Bàn cờ và các quân cờ có ích, nhưng chúng có thể bị lược bỏ. Điều quan trọng là hiểu được cách các quân cờ di chuyển và hành động. Thật vô nghĩa khi nói về ``định nghĩa'' hay ``bản chất thực sự'' của một quân tốt hay quân vua. Tương tự, hình học không quan tâm đến việc điểm và đường thẳng ``thực sự là gì''. Chúng vẫn là những khái niệm không được định nghĩa, và các tiên đề của hình học chỉ rõ mối quan hệ giữa chúng: hai điểm tạo thành một đường thẳng, v.v.. Đó là các qui tắc, và không có gì thiêng liêng về chúng. Các dạng hình học khác nhau dựa trên các tập hợp tiên đề khác nhau, và cấu trúc lí luận của hình học phi Euclid độc lập với mối liên hệ của nó với thực tế. Các nhà vật lí đã nghiên cứu chuyển động của các vật thể dưới các qui luật hấp dẫn khác với của Newton, và những nghiên cứu này có ý nghĩa ngay cả khi qui luật hấp dẫn của Newton được chấp nhận là đúng trong tự nhiên.}

\noindent\rule{\textwidth}{.4pt}

\noindent\rule{\textwidth}{.4pt}

\section*{(b) Nền tảng trực giác}

{\bf Trái ngược với cờ vua, các tiên đề của hình học và cơ học có nền tảng trực giác. Thực tế, trực giác hình học mạnh mẽ đến mức nó dễ dàng vượt qua suy luận logic. Mức độ mà logic, trực giác, và kinh nghiệm thực tế phụ thuộc lẫn nhau là một vấn đề mà chúng ta không cần phải đi sâu vào. Chắc chắn rằng trực giác có thể được rèn luyện và phát triển. Người mới học cờ vua sẽ di chuyển một cách thận trọng, nhớ lại từng quy tắc riêng lẻ, trong khi người chơi có kinh nghiệm có thể nắm bắt được một tình huống phức tạp chỉ qua cái nhìn thoáng qua và không thể giải thích hợp lý trực giác của mình. Tương tự, trực giác toán học phát triển cùng với kinh nghiệm, và có thể hình thành cảm giác tự nhiên với các khái niệm như không gian bốn chiều. Thậm chí, trực giác tập thể của loài người cũng có xu hướng tiến bộ. Các khái niệm về trường lực của Newton và tác động từ xa, hay khái niệm sóng điện từ của Maxwell, lúc đầu bị coi là ``không thể tưởng tượng được'' và ``trái với trực giác.'' Công nghệ hiện đại và đài phát thanh trong các hộ gia đình đã phổ biến những khái niệm này đến mức chúng trở thành một phần của từ vựng thông thường. Tương tự, sinh viên hiện đại không còn hiểu được những phương thức tư duy, thành kiến và các khó khăn khác mà lý thuyết xác suất phải vượt qua khi nó còn mới. Ngày nay, các tờ báo báo cáo về mẫu khảo sát ý kiến công chúng, và phép màu của thống kê bao trùm tất cả các khía cạnh của cuộc sống, đến mức những cô gái trẻ theo dõi thống kê về cơ hội kết hôn của mình. Do đó, mọi người đều có cảm nhận về ý nghĩa của những phát biểu như ``cơ hội là ba trong năm.'' Mặc dù còn mơ hồ, trực giác này đóng vai trò nền tảng và dẫn dắt bước đầu tiên. Nó sẽ được phát triển thêm khi lý thuyết tiến triển và có sự tiếp xúc với các ứng dụng tinh vi hơn.}


\end{document}
